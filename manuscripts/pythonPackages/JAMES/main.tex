%%%%%%%%%%%%%%%%%%%%%%%%%%%%%%%%%%%%%%%%%%%%%%%%%%%%%%%%%%%%%%%%%%%%%%%%%%%%
% AGUJournalTemplate.tex: this template file is for articles formatted with LaTeX
%
% This file includes commands and instructions
% given in the order necessary to produce a final output that will
% satisfy AGU requirements, including customized APA reference formatting.
%
% You may copy this file and give it your
% article name, and enter your text.
%
% guidelines and troubleshooting are here: 

%% To submit your paper:
\documentclass[draft]{agujournal2019}
\usepackage{url} %this package should fix any errors with URLs in refs.
\usepackage{lineno}
\usepackage[inline]{trackchanges} %for better track changes. finalnew option will compile document with changes incorporated.
\usepackage{soul}
\linenumbers
%%%%%%%
% As of 2018 we recommend use of the TrackChanges package to mark revisions.
% The trackchanges package adds five new LaTeX commands:
%
%  \note[editor]{The note}
%  \annote[editor]{Text to annotate}{The note}
%  \add[editor]{Text to add}
%  \remove[editor]{Text to remove}
%  \change[editor]{Text to remove}{Text to add}
%
% complete documentation is here: http://trackchanges.sourceforge.net/
%%%%%%%

\draftfalse

%% Enter journal name below.
%% Choose from this list of Journals:
%
% JGR: Atmospheres
% JGR: Biogeosciences
% JGR: Earth Surface
% JGR: Oceans
% JGR: Planets
% JGR: Solid Earth
% JGR: Space Physics
% Global Biogeochemical Cycles
% Geophysical Research Letters
% Paleoceanography and Paleoclimatology
% Radio Science
% Reviews of Geophysics
% Tectonics
% Space Weather
% Water Resources Research
% Geochemistry, Geophysics, Geosystems
% Journal of Advances in Modeling Earth Systems (JAMES)
% Earth's Future
% Earth and Space Science
% Geohealth
%
% ie, \journalname{Water Resources Research}

\usepackage{listings}
\usepackage{array}
\usepackage{amsmath}
\usepackage{amssymb}
\usepackage{amsfonts}
\usepackage{amscd}
\usepackage{tikz}
\usetikzlibrary{snakes,arrows,shapes}
\usepackage{graphicx}
%\usepackage{caption}
\usepackage{appendix}
\usepackage{hyperref}
\usepackage[utf8]{inputenc}
\usepackage[round]{natbib}
\usepackage{bm}
%\usepackage{chemist}
%\usepackage{chemformula}
\usepackage{breqn}
\usepackage{enumitem}
\usepackage{setspace} %needed for preprint in doublelinespace 
%\usepackage[colorinlistoftodos, textwidth=4cm, shadow]{todonotes} %comments and
\usepackage[normalem]{ulem} % provides strike out
\usepackage{caption}
\DeclareRobustCommand{\figref}[1]{ Fig. \ref{#1}}
\DeclareRobustCommand{\revref}[1]{Comment: \ref{#1}}
\DeclareRobustCommand{\Theoref}[1]{Theorem \ref{#1}}
\DeclareRobustCommand{\Defref}[1]{\mbox{Definition \ref{#1}}}
\DeclareRobustCommand{\enumref}[1]{\ref{#1}}
\DeclareRobustCommand{\appendixref}[1]{appendix \ref{#1}}
\DeclareRobustCommand{\Appendixref}[1]{Appendix \ref{#1}}
\DeclareRobustCommand{\sectionref}[1]{ section \ref{#1}}
\DeclareRobustCommand{\tupelref}[2]{\enumref{#1} \enumref{#2}}
%\definecolor{dark-green}{rgb}{0,0.25,0}
% vectors and matrices
\renewcommand{\vec}[1]{\boldsymbol{\mathbf{#1}}}
\newcommand{\tens}[1]{\mathrm{#1}}

% d/dt not italic, dt in integrals d upright, even putting the space before
\newcommand{\deriv}[1]{\frac{\mathrm{d}}{\mathrm{d}#1}}
\newcommand{\dd}[1]{\,\mathrm{d}#1}
\newcommand{\pderiv}[1]{\frac{\partial}{\partial #1}}

\newcommand{\suml}{\sum\limits}
\newcommand{\prodl}{\prod\limits}
\newcommand{\intl}{\int\limits}
\newcommand{\liml}{\lim\limits}
\newcommand{\LAPM}{\texttt{LAPM} }
\newcommand{\sympy}{\texttt{SymPy} }
\newcommand{\python}{\texttt{Python} }
\newcommand{\CompartmentalSystems}{\texttt{CompartmentalSystems} }
\newcommand{\ComputabilityGraphs}{\texttt{ComputabilityGraphs} }
\newcommand{\todo}[1]{{\color{red} comment:\\
#1} }
%\newcommand{\bgc_md2}{\texttt{bgc \textunderscore md2}}

% numbers
\newcommand{\N}{numbersystem:N}
\newcommand{\R}{numbersystem:R}
\newcommand{\CN}{numbersystem:C}
\newcommand{\Rpd}{numbersystem:Rpd}
\newcommand{\Rp}{numbersystem:Rp}

\newcommand{\tnot}{\gls{symb:t0}}
\newcommand{\transpose}{symb:transpose}


\graphicspath{{./figures/}}

\journalname{Journal of Advances in Modeling Earth Systems (JAMES)}


\begin{document}

%%%%%%%%%%%%%%%%%%%%%%%%%%%%%%%%%%%%%%%%%%%%%%%
% KEY POINTS
%%%%%%%%%%%%%%%%%%%%%%%%%%%%%%%%%%%%%%%%%%%%%%%
%  List up to three key points (at least one is required)
%  Key Points summarize the main points and conclusions of the article
%  Each must be 140 characters or fewer with no special characters or punctuation and must be complete sentences

% Example:
% \begin{keypoints}
% \item	List up to three key points (at least one is required)
% \item	Key Points summarize the main points and conclusions of the article
% \item	Each must be 140 characters or fewer with no special characters or punctuation and must be complete sentences
% \end{keypoints}

\begin{keypoints}
\item enter point 1 here
\item enter point 2 here
\item enter point 3 here
\end{keypoints}


\correspondingauthor{Markus Müller}{markus.mueller.1.g@googlemail.com}

%%%%%%%%%%%%%%%%%%%%%%%%%%%%%%%%%%%%%%%%%%%%%%%
%  TITLE
%
% (A title should be specific, informative, and brief. Use abbreviations only
% if they are defined in the abstract. Titles that start with general keywords
% then specific terms are optimized in searches)
%
%%%%%%%%%%%%%%%%%%%%%%%%%%%%%%%%%%%%%%%%%%%%%%%

% Example: \title{This is a test title}
\title{The biogeochemichal model database \texttt{bgc\_md2} and python
packages  \LAPM, \CompartmentalSystems, \ComputabilityGraphs{} for the analysis of compartmental dynamical systems}
%\date{\today}
%%%%%%%%%%%%%%%%%%%%%%%%%%%%%%%%%%%%%%%%%%%%%%%
%
%  AUTHORS AND AFFILIATIONS
%
%%%%%%%%%%%%%%%%%%%%%%%%%%%%%%%%%%%%%%%%%%%%%%%

% Authors are individuals who have significantly contributed to the
% research and preparation of the article. Group authors are allowed, if
% each author in the group is separately identified in an appendix.)

% List authors by first name or initial followed by last name and
% separated by commas. Use \affil{} to number affiliations, and
% \thanks{} for author notes.
% Additional author notes should be indicated with \thanks{} (for
% example, for current addresses).

% Example: \authors{A. B. Author\affil{1}\thanks{Current address, Antartica}, B. C. Author\affil{2,3}, and D. E.
% Author\affil{3,4}\thanks{Also funded by Monsanto.}}

% \affiliation{1}{First Affiliation}
% \affiliation{2}{Second Affiliation}
% \affiliation{3}{Third Affiliation}
% \affiliation{4}{Fourth Affiliation}

%\affiliation{=number=}{=Affiliation Address=}
%(repeat as many times as is necessary)

\author[2]{M{arkus M{\"{u}}ller}}
\author[4]{Holger Metzler}
\author[3]{Ver{\'{o}}nika Ceballos-N{\'{u}}{\~{n}}ez}
\author[2]{Kostiantin Viatkin}
\author[1]{Carlos A. Sierra}
\author[2]{Yiqi Luo}
\affil[1]{Max Planck Institute for Biogeochemistry, Hans-Knöll-Str. 10, 07745 Jena, Germany}
\affil[2]{Cornell University, 616 Thurston Ave. Ithaca, New York 14853, USA}
\affil[3]{}
\affil[4]{}

\begin{abstract} \noindent
  
Compartmental systems are a particular class of dynamical systems that describe
the flow of conserved quantities such as mass and energy through a network of
interconnected compartments.  The main purpose and greatest challenge of this
work is to make such models {\bf transparent} by  facilitating  comparisons
between them.  Firstly the scientific challenge is to find the common
diagnostics by which we can compare models. The technical challenge is how to
implement them in a way to make them applicable to many models and thirdly how
to describe models compactly enough to facilitate a maintainable collection of
reasonable size and flexible enough to formulate models in the many different
ways in which they appear in the literature.  To address the first question we
enhance the common diagnostics of pool contents and fluxes by the the
computations of age and transit time distributions for whole models and
especially subsystems like vegation or soil which is essential to be able to
compare models with conceptually different pools, i.e. two ecosystem models
will likely both have a vegetation part , however consist it might consist of two pools
(i.e. leaf and root) in one but three pools (i.e. leaf, wood, root) in the
other.  
To address the technical challenge we avoid duplication in two very
rigorous and novel ways: We provide common eqivalent building blocks for models
implemented in symbolic math via \texttt{sympy} as special types, and a set of
type annotated functions operating on these types as building blocks which can
be combined by a graph library to compute every possible result,
reachable by any recursive combination of these functions. 
This allows us to formulate models compactly and
flexibly in different equivalent ways i.e. via fluxes, flowdiagrams, or
matrices and at the same time to avoid the implementation of many of similar
functions for the same result.  Thus we can also not only query and compare
what is provided, as in the model description records of a conventional data
base, but also what is {\bf computable} from them.  The combination of these
capabilities is expressed in the package \texttt{bgc\_md2} (Biogeochemical
Model Database ), which apart from the tools mentioned above also represents a small
(30+) collection of models, motivated by our own work on the terrestrial carbon
cycle.
As a proof of concept of this approach we use \texttt{bgc\_md2} to reconstruct
four models of the trendy9 model intercomparison, symbolicly, guess some parameters
run them with the trendy9 driver data and compare them with respect to transient 
mean ages and transit times of carbon trough the vegetation and soil subsystems.
This presentation concentrates on the software and only links to the mathematical foundations
which are published already or in preparation.

%implemented analysis tools include symbolic computations  of model
%decomposition into subsystems, flowdiagrams, transformation and reformulation
%with respect to different building blocks e.g. matrix versus flux equations and
%also difficult to obtain numerical metrics to characterize timescales of mass
%flow in compartmental systems such as the age of the mass and transit time
%trough the entire system, subsystems like vegetation or soil, or individual
%compartments.
%The actual computations were outsourced into other packages where \LAPM (Linear
%Autonomous Pool Models) provides functions for the analysis of compartmental
%systems at equilibrium, \CompartmentalSystems tools for the analysis of
%non-autonomous linear compartmental systems and \ComputabilityGraphs
%determines the set of all possible computations that can be performed on a
%model depending on information available from its description.

\end{abstract}

\section*{Plain Language Summary} 
Imagine a bucket standing on a table under a water tap. Now drill some small
holes in the bottom and attach some short pipes that lead to other buckets
standing on some chairs and drill holes in their bottom to attach pipes to even
more buckets standing on the ground. Drill holes again in the bottom so that
water can leak out...  It turns out that regardless of the number of buckets
pipes or taps, the amount of water in the buckets at any time in the future can
be predicted very accurately without knowing the details of how exactly the
water flows through the buckets, for instance where the holes in the bottom are
or which path a single water molecule takes through the bucket.  Actually it is
enough to know three things: how much water is in the buckets at the start, how
much water comes out of the tap at any time and how the outflux through the
holes depends on the amount of water in the bucket i.e. how big the holes are.
The simplicity of this description is extremely attractive and has made bucket
or pool models prolific in many branches of science, describing, apart from the
obvious physical examples, phenomena including  chemical reactors or
ecological earth system models for the carbon cycle, essential for estimating
future climate change.  All these models are called \emph{compartmental
models}.  We developed a set of software tools that makes it much easier to
describe, translate, collect, inspect, and ultimately compare such models.
While the tools are as widely applicable as compartmental models, the example
applications we present are inspired by our own work in the global carbon cycle
where the reduction of uncertainty of model predictions is a major goal for
climate change mitigation and a strong motivation for model inter comparisons
and the increased transparency facilitating them.  Our tools can translate the
simple description given above into a system of ordinary differential equations
and vice versa visualize such a matrix  equation as system of interconnected
buckets. They can compute complex numerical results that can in principle
be compared to measurable physical quantities.  The ability to do this for a
large number of models, made possible by a parsimonious description, is
unprecedented and makes every single model much more transparent.  Our software
tools are free to use, change and extend.  The aim of this article is to make
them available to researchers who want to use them, show the way how they can
be extended but also to discuss what more fundamental lessons we have learned
in the process of their creation e.g. w.r.t which criteria compartmental models
should be compared.
%%%%%%%%%%%%%%%%%%%%%%%%%%%%%%%%%%%%%%%%%%%%%%%
%
%  BODY TEXT
%
%%%%%%%%%%%%%%%%%%%%%%%%%%%%%%%%%%%%%%%%%%%%%%%

%%% Suggested section heads:
% \section{Introduction}
%
% The main text should start with an introduction. Except for short
% manuscripts (such as comments and replies), the text should be divided
% into sections, each with its own heading.

% Headings should be sentence fragments and do not begin with a
% lowercase letter or number. Examples of good headings are:

% \section{Materials and Methods}
% Here is text on Materials and Methods.
%
% \subsection{A descriptive heading about methods}
% More about Methods.
%
% \section{Data} (Or section title might be a descriptive heading about data)
%
% \section{Results} (Or section title might be a descriptive heading about the
% results)
%
% \section{Conclusions}


%\section{Motivation and significance}
\section{Introduction}
The principle of mass conservation plays a central role in mathematical models
of natural systems in a variety of scientific fields such as systems biology,
toxicology, pharmacokinetics \cite{Anderson1983} , ecology
\cite{Eriksson1971ARoEaS, Rodhe1979Tellus, Matis1979, Manzoni2009SBB},
hydrology \cite{Nash1957IASH, Botter2011GRL, Harman2014GRL}, biogeochemistry
\cite{Manzoni2009SBB, Sierra2015EM}, and epidemiology \cite{Jacquez1993SIAM}.
In most cases such models are non negative dynamical systems that can be
described by systems of ordinary differential equations (ODEs) with
strong structural constraints.  Such systems are called compartmental systems
\cite{Anderson1983, Jacquez1993SIAM, Walter1999, Haddad2010}
, and have
important mathematical properties that aid in their analysis and study.
\kv{Markus}{I removed ``first order'' from the paragraph above - the features discussed in this paper are not limited to first-order models, are they?}\\
\mm{Kostia}{All equations are first order in the mathematical sense: only first derivatives occur. 
(That does not mean they are linear or have ``first order kinetics''... which is not a mathematical but rather an (unnecessary) ecological definition. ``linear'' would have been sufficient.) 
}
\subsection{Diagnostics of age and persistance} 
\kv{Markus}{I rephrased a few things, including the subtitle }
Characterizing dynamics of compartmental systems often includes content of the pools,
and the fluxes between the pools as functions of time.
As mass moves across a compartmental system, it is often also of interest to study
properties like the \emph{age} of mass in a pool, the entire system
or a subsystem, and the \emph{transit time} of mass 
across the entire network of compartments or parts of it until its
final exit. These diagnostics are especially important for the studies of biogeochemical cycles 
related to persistence of carbon or nutrients in ecosystems \cite{schmidt_persistence_2011, friedlingstein_uncertainties_2014, woolf_microbial_2019, kyker-snowman_stoichiometrically_2020}
As we will see later for compartmental systems ages and transit times are characterized by
distributions that are time dependent. 
However, methods traditionally used in the literature, to compute these metrics for
compartmental systems depended on specific assumptions imposed on the system
of equations \cite{Sierra2017GCB} and restricted either the applicability 
of the procedures or the interpretability of the results if applied to situations 
where those conditions where not fulfilled exactly. 
\mm{Kostia}{
  The following sentences are important for the ``red thread'' (narrativ) of the paper that stresses 
  \emph{automated} and therefore \emph{implementable} computability over vague verbal statements.  
  I rephrased ahd moved them here (You had deleted them from the next chapter.)
}
This becomes a challenge when they are implemented in software
-particularly if this software is a data base that needs to accept \emph{queries}. The result of a query must be implemented by a function and therefore unambigiously defined. If the conditions are not met the misapplication of the function must be preventable \emph{automatically}.
As we will show, we fulfill this much stricter requirement which is much more difficult to achieve than a verbal warning about the limits of an approximation or it's interpretation.


\subsection{Analysis and Intercomparizon of Multiple Models}

Development of compartmental models for studying ecological processes has
greatly increased in the recent decades \cite{le_noe_soil_2023,
chandel_microbial_2023, MANZONI20091355}.  This presented the researches with
opportunities to analyse the differences between different model structures,
compare the assumptions behind them, and their impact on the predictions.  In
terms of analysing and comparing multiple models, two main approaches have been
used so far.  First approach would entail a single modelling group implementing
several models, parametrising and running them with specific drivers, then
analysing the results based on metrics implemented in the modelling
``testbed'' \cite{WalkerMultiAssumption2018, SulmanMultiple2018,
LiaoDisentangling2022}. The main benefit of this approach is that it gives
researchers full control over the models and flexibility to implement and
analyse them in any preferred way. However, such efforts usually include a
limited number of models, and a limited scope - both in in terms of geograophy,
and in terms of inclusion of biogeochemical processes.  The reasons for this
are the time and technical complexity needed to implement complex models and
their diagnostics in the code. Hense, the setup of such model testbeds is
usually project-specific, and the attempt to use them for another study with
different model composition and different diagnostics would usually require a
substaintial code re-writing.  On the other hand, projects that analyse a large
number of complex global models \cite{sitch_recent_2015, eyring_overview_2016,
collier_international_2018} would use the second approach, where different
modelling groups take care of running their own models based on prescribed
drivers, and deliver agreed-upon outputs, which are subsequently analysed.
While this approach allows distributing the responsibility of handling the
coding, it considerably limits the ways in which models can be jointly analysed
- comparison is limited to certain generalised model outputs.  The package we
present here aims at facilitating the benefits of both approaches, i.e.
flexibility in the choice of model diagnostic, parametrizations and drivers,
distributed coding, while at the same time minimizing the drawbacks, i.e.
complexity of coding, incompatibility between codes, difficulty in reproducing
and expanding modelling studies.

Over the course of more than ten years and with the help of contributions of
several people we have developed  practically usable tools to create, inspect
and query a collection of compartmental models, i.e. a model data base.  We
have learned some hard lessons in the process.  While coding has become a major
part of science, and greatly enhances our ability of prediction, it has also
obscured the ways to obtain them.  A way to apply the \emph{same} tools to
\emph{many} models is a very attractive proposition to regain some of the
transparency by well defined comparisons.  As the number of models and the
number of applicable diagnostics grow the the number of combinations grows
bilinearly, and with it the technical challenges to maintain both collections.
A 'copy and paste' approach becomes unmaintainable very quickly, in our case
definitely before the present number of about 30+ models had been reached.
Ideally we want to turn the bilinear growth of combinations of models and
diagnostic tools into an advantage and make all diagnostics, and especially new ones, instantly
available to all models.  To make this possible the right abstractions have to be found, just to
keep the size of the code base in check.  At the outset it is very easy to
overlook that the right abstractions are an emerging feature and \emph{change}
with the addition of more models and new methods with unforeseen properties.  To
be able to repeatedly refactor the code base w.r.t better abstractions requires
software engineering techniques like unit testing and more specifically a 'test
hull' i.e. a comprehensive set of tests to provide coverage for the
features, so that their implementation can be changed with confidence.  Apart
from the benefit of inducing better engeneering practices the work on a model
data base has the benefit of \emph{unveiling} these abstractions.  Their choice
is not accidental but the result of a decade of refinement and a major outcome
of the project.  An example for this clarifying feedback of software
development to science is  the basic requirement, to be able to  define  a
result of a computation by a function implementing this computation.
Astonishingly this excludes some metrics for compartmental systems that have
been proposed, as we will see.

Our first encounter with the challenge of a model collection was the
development of \SoilR{} \cite{Sierra2012GeosciModelDev}, a \texttt{R} package
that collects a number of compartmental models specialized on soil models.  The
software we are going to present can be seen as the result of countless
refactoring and two major design iterations of this initial project, the result
of contributions of many codevelopers, user input,  and a decade of critical
evaluation and resulting change.  The result not only far
exceeds \SoilR's capabilities but also constitutes our best proposal for an
extendable community tool so far.  In particular the new approach
\begin{enumerate}
  \item
    is based on symbolic math, which allows users to look at models
    as they appear in publications, in the form of equations and flow diagrams,
    and enables symbolic analysis (e.g. computing derivatives for sensitivity analysis)
  \item 
    describes compartmental models as graphs, \emph{or} matrix equations and enables the \emph{automatic conversions} between them, thereby enable ling the definition of subsystems by simply naming their pools.
  \item
    adds diagnostics (transit time and age distributions) not present in \SoilR{}.
  \item
    provides a much simpler user interface 
    to the vast number of analytical tools that had already made the creation of a navigable documentation an overwhelming job for \SoilR. 

  \item
    simplifies the description of models by much smaller building blocks that are combined
    \emph{automatically}, and combination of few or many such building blocks into a `model record',
    rather than forcing the developers to implement (and a user to choose) a
    huge number of model constructing functions.
  \item 
    \emph{automatically combines functions} to compute results recursively
\end{enumerate}

We developed three python packages for the representation, classification, collection and analysis  of compartmental systems, implementing methods for the computation of age and transit times, that can be performed for particular systems under given assumptions, and tools to automatically \emph{restrict} their use.
These open source packages are called: 
\LAPM{} (Linear Autonomous Pool Models),
\CompartmentalSystems{}  and
\texttt{bgc\_md2} (Biogeochemical Model Database).
This manuscript provides a general introduction to these packages 
with emphasis of their combined use via \texttt{bgc\_md2}.
The latter implements a set of model describing properties that form the (exclusive) arguments and return values of strictly typed functions, forming a network of computability which we use 
to implement a declarative domain specific language for model creation, inspection, querying and thus comparison.
To facilitate this we implemented a fourth package \ComputabilityGraphs{} based upon the  type hinting syntax of modern \python{} versions, extending its use to predicting what is computable.   
The requirement to define every model property as a variable of a specific
type, and to provide a function to compute it from other model property (i.e.
variable of well defined \emph{types}) also provides a rigorous filter for what
constitutes a diagnostic variable or model property. 
We use it as such to discuss why we implemented or excluded some diagnostics proposed in the literature.
Practically we use this language to create a small
(30+) collection of models, motivated by our own work on the terrestrial carbon
cycle.  Some screenshots of the software are shown in \figref{fig:overview}. 
\kv{Markus}{ Figure reference didn't work here}
\mm{Kostia}{To make them work  you have to  run pdflatex twice (first it gathers the references and then uses them) or use the ``Makefile'' by just saying make (I don't know though if your git shell on windows has make installed):
The manual version to create a new pdf with updated references and citations is the following sequence:
}
\begin{minted}{bash}
pdflatex main.tex
bibtex main.aux
pdflatex main.tex
pdflatex main.tex
\end{minted}
\mm{Kostia}{sometimes (when the references are messed up you have to remove the helper files by ``make clean'' or manually:} 
\begin{minted}{bash}
rm *.aux *.bbl *.blg
\end{minted}
\mm{Kostia}{It's also good practice to run pdflatex main.tex often (at least before checking in) so that small errors are found quickly.
It's much harder to erase them later, since latex error messages are sometimes not very helpful. }


\begin{figure}[h]
\includegraphics[width=\columnwidth]{TabScreenCombined.pdf}
  \caption{
      Figure description, top row, left to right: Interactive Jupiter widget
      with a table of models (orange buttons can be clicked to expand or
      collapse a more detailed view of the particular model), Model inspection
      with pool connection graph, which can be derived from the symbolic
      description along other symbolic properties as flux equations and the
      compartmental matrix, Zoom into IPython/Jupyter UI, showing methods
      automatically added by the computability graph library.  \\ Bottom row:
      Data assimilation with an automatically created numeric model (from
      symbolic description), Computability graph for a desired diagnostic
      (aggregated Flux from the vegetation to soil part, showing the
      additionally needed information to compute the desired result)
  }
  \label{fig:overview}
\end{figure}

%\subsection{Examples}
As a conceptual proof of the  approach implemented by \texttt{bgc\_md2}
we demonstrate it on the example of four biogeochemical models from the TRENDYv9 \cite{GCB2020} model intercomparison project, that we reconstructed in a simplified form from their description in the literature.
We express the models symbolically, provide some parameters run them with some TRENDYv9 driver data and
compare them with respect to transient mean ages and transit times of carbon
trough the vegetation and soil subsystems.  
While some of the implemented algorithms for compartmental subsystems are novel 
and yet unpublished, this work focuses on the algorithmic and structural proposal for a model collection
i.e. a model data base to facilitate their use for model intercomparison. 
We provide examples based on the global carbon cycle, but the
packages can be used for a large variety of systems in which mass or energy
conservation is required.
The remainder of the article is structured as follows.
In \sectionref{sec:ConceptualFramework} we briefly describe compartmental systems, with some of the details
relegated to \appendixref{appendix:MatrixDerivation}.
We then provide an overview over the main role of the implemented packages in 
\sectionref{sec:PythonPackages}. A brief introduction to example applications
is given in \sectionref{sec:ExampleApplications} followed by a discussion of
possible present and future use cases in
\sectionref{sec:ConclusionsAndOutlook}. \todo{@Markus for some reason, this and previous section references didn't work }




%:
\section{Conceptual framework}
\label{sec:ConceptualFramework}
\subsection{Definition and classification of compartmental systems} 

The most parsimonious and general description of a compartmental system is a graph in
the mathematical sense, which is a tuple of two sets, the set of nodes, and the
set of edges.  In the particular graphs that describe compartmental systems the
compartments (or pools) form the nodes and the fluxes between them and the
exterior the edges.  The fluxes are allowed to be functions of time and the
contents of the pools exclusively. This is usually referred to as `well mixed' or
`kinetically homogeneous' compartments.  

If we assume any arbitrary ordering of the pools we can represent the mass (content of the pools)
by an ordered tuple $\vec{x}$.
Because mass is a non-negative quantity, this vector of mass contents can
only occupy the non-negative orthant of the state-space; i.e. $x \in
\mathbb{R}^n_+$. 
Likewise the mass, that the system receives from outside is represented
by a tuple of mass input fluxes (mass over time) $u \in \mathbb{R}^n_+$.
It has been shown in \cite{Jacquez1993SIAM}
that for smooth enough fluxes, mass is transferred among compartments and released back to the external
environment can be according to rates encoded in a compartmental matrix $B \in
\mathbb{R}^{n \times n}$. 
Therefore, we can write the dynamics of a
compartmental system as a set of ordinary differential equations of the form
\begin{equation} \label{eq:CompartmentalSystem}
\dot{\X} = \frac{d\X}{dt} = \I(\X, t) + M(\X, t) \, \X
\end{equation}
The key property of compartmental systems is, in order for the system to balance mass, that the square matrix $M=(M_{ij})$ exhibits three main properties
\begin{enumerate}
  \item $M_{ii}\leq0\text{ for all }i$,
  \item $M_{ij}\geq0\text{ for all }i\neq j$, and
  \item $\suml_i M_{ij}\leq 0\text{ for all }j$.
\end{enumerate}
{\todo @Markus Please add the description of notations you use in formulas throughout the paper}
Then, $\tens{M}$ is called \emph{compartmental} and governs all internal cycling of material as well as the exit of material from the system.
We describe the relationship between the graph and matrix based descriptions in \appendixref{appendix:MatrixDerivation}.

We distinguish between different types of compartmental systems, according to linearity and autonomy. If the vector of inputs and the compartmental matrix depend on the vector of states in system \eqref{eq:CompartmentalSystem}, we call it non-linear, and linear otherwise. Similarly, if the vector of inputs and the compartmental matrix depend on time, we call the system non-autonomous, and autonomous otherwise. 
Both descriptions of compartmental system, the intuitive graph of pools and fluxes and the 
matrix based formulation as an ODE system have advantages for some applications. 
While the graph based description allows for composition of subgraphs and decomposition of models into e.g vegetation and soil part, the ODE description facilitates the description and implementation of some advanced numerical algorithms as explained below.
In our approach both formulations are converted into each other automatically as need arises.
This effortless switching between the viewpoints is the main facilitator of the new level of transparency of our approach, their combination enables the computation of transit time and mean age systems for subsystems of selected pools with no more effort than defining the pools constituting them.

\subsection{System and sub-system level metrics: contents, fluxes, age, transit time}
Compartmental systems can be described by a set of building blocks and metrics. 
Scientific inference regarding modelling results, as well as model evaluation and intercomparison, often depends on the metrics that are used to describe model output.
Depending on the chosen metric, results can be both informative of model peformance and useful for a real-world question of the study, or, they can be overly reliant on artificial assumptions and therefore uninformative for a study of natural, rather than modelling, mechanisms.
Hense, we consider a scientific discussion on what is to be regarded as a proper metric to be particularly relevant.
In this study we propose the following four criteria for compartmental system metrics:
\begin{enumerate}
  \item
    \label{enum:function}
    In our definition we require that a metric must be unambiguously computable by a function of other metrics or building blocks of a model. 
  \item
    \label{enum:general}
    Preferably metrics should be applicable to all (nonautonomous, nonlinear) compartmental systems, i.e. not rely on e.g. equilibrium conditions.
  \item
    \label{enum:measurable}
  Preferably metrics should also be physical properties
  i.e. have a clearly defined way to measure them at least in principle.
  \item
    \label{enum:meaningful}
  Preferably metrics should be relevant to answer research questions related to real-world phenomena
\end{enumerate}

The first criterium can be implemented through the formulation of metrics as return values of
strictly typed functions. We will show later that this makes the creation, inspection and
comparison of compartmental systems much more feasible by software tools. 

The commonly used metrics to characterise compartmental system behavoir are the
contents of the pools and fluxes between them. While they are general and
unambiguousely defined, the last two criteria - ability to measure and
meaningfully interpret them - depend on model formulation. For example, a
photosynthesis flux can in principle be measured \textit{in-situ} or estimate
from remote sensing \cite{jez_emerging_2021, sun_remotely_2023}, however soil
carbon pools and fluxes are conceptualized quite differently in existing
models, in many cases without a possibility to measure them or a clear physical interpretation 
%\cite{abramoff_millennial_2018}
.  In the latter situation,
aggregated pools or fluxes become more interpretable, e.g. total soil carbon,
or a litter decomposition flux, hense to satisfy all the criteria we may need
to define a sub-system for which a metric is meaningful.   

Beside the content of pools the whole system or subsystem and fluxes between
them, ages, and transit times are key quantities of compartmental systems that
fulfill all of these criteria. While age describes how old material in the system is, transit time describes how
long material needs to travel through the entire system from entry to exit
\cite{bolin1973Tellus, Sierra2017GCB} 
\todo {@Markus something didn't work here with enumref} %: \enumref{enum:function}, \enumref{enum:general} and \enumref{enum:measurable}.
Both age and transit time distributions can be unambiguosely computed for every pool at every time point, given inital distributions at the start of the simulation.
They are applicable to general nonautonomous nonlinear systems, and help to better understand underlying system dynamics and to compare models
with different sizes or structures. They can in principle be measured: radiosotopic studies have shown the ways to measure age distributions e.g. of carbon in soils, and to constrain models  \cite{Trumbore2016, shi_age_2020} 

And finally, they are very relevant for the studies of biogeochemical cycles, e.g. looking at the ability of soils to retain carbon and nutrients in such contexts as soil health or climate change. 
This makes them the first choice for model
comparisons. 


There are metrics that are not suitable to compare \emph{all} compartmental systems because they make further assumptions, e.g. the existence of equilibria. 
General compartmental systems can be nonlinear and nonautonomous, in the latter case making the concept of equilibrium itself meaningless and in the former the computation of the possibly empty set of fixed points by far to difficult to attempt (by a function).
If we wanted to include a strictly typed function to do it, its argument would
have to be of a new type for a linear autonomous system.  Achieving the same flexibility
and complete transparency w.r.t. flux, rate, or matrix centered description
will necessitate additional types for constant rates or linear fluxes.
We note that metrics build upon equilibrium assumptions fail to meet criteria \enumref{enum:general}

Another class that fails at least one of the criteria
are properties that can still be unambiguously defined by a function and
therefore computed but are not physical quantities and are therefore to be
considered properties of the model rather than reality. 
\footnote{
  That a flux `rate' should be considered a model property (or latent variable)
  rather than a physical property becomes more apparent when we consider
  processes that treat material of different age or position in the pool
  differently.  In this case the assumption that the material in the pools is
  `well mixed', and can therefore be treated by the \emph{same} exit `rate'
  would be violated, and the model no longer 'compartmental.  However such
  processes clearly exist and fluxes and mass are still measurable.  The
  concept of a single 'rate' applied to all material in a pool is implied by
  the definition of compartmental systems.
} 
Flux rates and compartmental matrices, and more importantly (some) properties derived from them, fall into  this category. 
One example is a commonly used metric `turnover time' which is defined as the quotient between content of and outflux from a pool, or as the inverse of the rate. 
`Turnover time' is an interesting example as it violates either, \enumref{enum:measurable} or \enumref{enum:general}, 
because it turns out that for a one pool system in equilibrium it actually coincides with the mean backward transit time
which could be approximated by many measurements but loses this interpretation in any non-equilibrium state \cite{Sierra2016GlobChangBiol, Lu2018Biogeosciences}.

More recently proposed way to compute `residence time' through inverting a compartmental matrix \cite{Xia2013GCB, Luo2017Biogeosciences, Luo2022, luo_land_2022} suffers from the same issue \cite{Sierra2016GlobChangBiol}  
Consequently, other proposed metrics derived from the equilibrium assumption, such as `carbon storage capacity', `carbon storage potential' \cite{Luo2017Biogeosciences, Huang2017GCB, zhou_traceability_2021, LiaoDisentangling2022, Luo2022, luo_land_2022},
have a physical meaningful interpretation only in equilibrium but lose it if their formal definition is extended to the transient case (nonautonomous compartmental systems) \cite{Sierra2018JAMES}.

While we could implement a function applying such a definition, it would not be general, and therefore its interpretation would be project-specific.
In particular, equilibrium-based metrics would have different interpretations when applied to an assumed steady state system as in \cite{Xia2013GCB}, or to an approximation of a transient system as a sequence of quazi-equilibria as in
\cite{Koven2015Biogeosciences, osullivan_process-oriented_2022}, and hardly any meaningful interpretation when the study is focused on the transient nature of the system \textit{per se}, as in \cite{Luo2017Biogeosciences} 
In the latter case, a user asking for 'Residence Time' (by applying such a function) would reasonably expect the transient 'time of residence' of material, but would actually receive the time of residence the material  
\emph{would} have \emph{if} the system was frozen at this moment \emph{and} allowed (infinite) time to reach its equilibrium. Therefore, to avoid possible misleading interpretations, in our package we only implemented metrics 
that are general and unambiguous, thus leaving it for the users to, if necessary, implement and justify additional assumptions or approximations used for their metrics. 

\todo{@Holger\\
In the following paragraph I played it save. Carlos had mentioned entropy in an old draft, but I don't know if the restriction to autonomous is a feature of the theorie or of \LAPM{}
}
A similar restriction of generality applies to applications of Shannon's information entropy as a complexity measure 
of dynamical systems \cite{Ebeling1998}. 
For 
%non? 
autonomous linear systems it has 
been used to describe the uncertainty of a particle's path through a
compartmental system, quantifying how difficult it is to predict this path. 
and to compare path properties of models with different number of
compartments and connections among them \cite{Metzler2020}. 
It's use can be extended to systems in equilibrium.

Surprisingly the literature even contains proposals for metrics that can not be defined as functions because they are ambiguously defined like the following matrix factorization approach.  
It has been claimed that for most Carbon cycle models the linear version of \eqref{eq:CompartmentalSystem} can be written in product form 
\cite{Xia2013GCB, Luo2017Biogeosciences, zhou_sources_2018, zhou_traceability_2021, LiaoDisentangling2022,Luo2022, luo_land_2022}.
\begin{eqnarray} 
  \frac{d\X}{dt} &=& \I(t) + M(t) \, \X 
  \label{eq:LinCompartmentalSystem} 
  \\
                &=& \I(t) + A \xi K(t) \, \X
  \label{eq:AxiK}
\end{eqnarray}

While it is certainly possible to derive \eqref{eq:LinCompartmentalSystem} unambiguously from \eqref{eq:AxiK} the opposite direction is not possible. Neither is the form \eqref{eq:AxiK} as general as \eqref{eq:LinCompartmentalSystem} 
nor is the decomposition uniquely defined.
While the issue with generality could be solved similarly to the equilibrium
issue by introducing subtypes of decomposable matrices and fluxes the issue of
ambiguity prevents the implementation of a function in the mathematical sense, which requires the return value to be unambiguously defined by the arguments.
This actually shows that any analysis built upon the decomposition is not
suited to discuss compartmental systems in general, not even if they can be
written in the form or \eqref{eq:AxiK}. This is relevant since such results have been published without discussion of the inherent ambiguity. We show in detail why this is the case in \appendixref{appendix:MatrixDerivation}.

todo{@Markus In the following paragraph the 1st sentence doesn't flow...rephrase a bit? Besides, I'm not sure all what you list axtually fulfills all th 3 criteria. E.g. contents of pools and fluxes between the pools may not be measurable at all,
especially if they are defined as \"fast\", \"medium\", \"slow\" soil pools. Maybe we need to find more some precise wording here.}
To summarize, among the traditionally proposed 
metrics for compartmental systems fulfill our conditions of unambiguous definition, general applicability to nonautonomous nonlinear, physical interpretability as principally measurable, and meaningful interpretability in the context of the study. 
The few that do are:
contents of pools, the system or subsystems, the fluxes into, between or out of them and the age and transit times distributions for the whole system or parts of it.  
To provide those metrics for all models is therefore a priority that is reflected by the structure of the project:
While we provide tools for many special cases and specialized results in the depth of our libraries, the most general results
are made much more accessible.

\subsection{Compartmental systems in equilibrium} \label{sec:Equilibrium}
The concept of equilibrium is restricted to autonomous systems. 
It does not even make sense to ask the question otherwise. \todo{@Markus quazi-equilibria as approximations? Can we mention something about that?}
If the autonomous system is nonlinear it is possible but not certain that an
equilibrium exists. The only case where we can expect an equilibrium are
linear, autonomous, pool models, 
\begin{equation} \label{eq:LS}
\dot{\X} = \I + M \, \X, \qquad  \mathrm{with} \quad \X(t_0) = \X_0.
\end{equation}
for which interesting properties can be
obtained by the \LAPM{} package.  The equilibrium $x^*$ is defined by the
condition  $\dot{\X}=0$ which translates to $-M \X^*= \I$ which means that for
pool contents $\X^*$ the influxes $\I$ match the outfluxes $M \X^*$ exactly.  It is
straightforward to see that for this to happen all 
pools with influxes must be connected (possibly via other pools) to an outflux of
the system, and that the (constant) rates for all the flux rates out of all
pools along this paths are greater than zero, since an input receiving pool $p$
without these conditions would necessarily grow over time, violation the
equilibrium condition $\dot{\X}_p=0$. Interestingly these conditions also
guarantee that $M$ is invertible and the equilibrium therefore uniquely
determined by: 
\begin{equation} 
\label{eq:x_Minv_I}
\X^* = - {M^*}^{-1} \, \I^*.  
\end{equation}
Although at different times different material moves through the system, the
size of the pools does not depend on time if the system is in
equilibrium: $\X(t)=\X^*$ 
This is also true for other properties such as the
age distribution of mass in particular compartments and in the entire system
and the transit time distribution, which is defined as the time it takes masses
in the input flux to appear in the output flux. 
Although the material moving through the system does change the amount, age and transit
time distributions do not. 
They are in fact characterized by the Phase Type
distribution, which depends on compartmental matrix $B$ and the equilibrium
solution $x^*$ for the system age distribution and the $B$ and the input $u$
for the transit time distribution \cite{Metzler2018MGS}.
Compartmental systems at equilibrium have similar properties as continuous-time absorbing Markov Chains \cite{Metzler2018MGS}.
Therefore, we can obtain other quantities of interest such as  the path entropy of particles that travel across the system and the occupation time of particles inside compartments \cite{Metzler2020}. These properties of linear autonomous compartmental systems at equilibrium can be obtained with the \LAPM{} package.

Interestingly the properties of linear autonomous systems in equilibrium can
also be computed for nonlinear systems in equilibrium if such an equilibrium
exists.
\begin{equation} \label{eq:NLS}
\dot{\X} = \I(\X) + M(\X) \, \X, \qquad  \mathrm{with} \quad \X(t_0) = \X_0,
\end{equation}
In equilibrium the system is indistinguishable from a linear autonomous one
\begin{align} 
  0= \dot{\X} = \I^* + M^* \, \X^*
\end{align}
with $M^*=M(\X^*)$ and $\I^*=\I(\X^*)$ 
If the inverse ${M^*}^{-1}$ exists transit time and age distributions can be computed 
\cite{Metzler2018MGS}.
and therefore also  nonlinear autonomous
systems at equilibrium can be analyzed with the by the
\LAPM{} package.
Note however,that 
\eqref{eq:x_Minv_I} is useless to determine $\X^*$ and 
no such $\X^*$ might exist for some systems
while others may have multiple fixed points and the age and
transit time distribution may be very different for these different equilibria.

%%%%%%%%%%%%%%%%%%%%%%%%%%%%%%%%%%%%%%%%%%%%%%%%%%%%%%%%%%%%%%%%%%%%%%%%%%%%
\subsection{Time Evolution} \label{sec:trajectory}
We consider now linear non-autonomous systems of the form
\begin{equation} \label{eq:NALS}
\dot{\X}(t) = \I(t) + M(t) \, \X, \qquad  \mathrm{with} \quad \X(t_0) = \X_0.
\end{equation}
In this case, the inputs and the compartmental matrix are time-dependent and
the system never converges to a fixed-point solution. In most cases, an
analytical solution cannot be obtained, but the solution can be 
obtained numerically. In particular, the solution for systems of the form of equation \eqref{eq:NALS} can be written as
\begin{equation}
(t, t_0, \X_0) = \Phi(t, t_0) \X_0 + \int_{t_0}^{t} \Phi(t, \tau) \I(\tau) \mathrm{d}\tau.
  \label{eqn:Phi}
\end{equation}
The state transition operator $\Phi(t,t_0)$ is a matrix-valued function that
multiplied with the state $x_0$ at $t_0$ transitions it to the state $x(t)$
subsequent time $t > t_0$. It is numerically computable by solving an matrix
ode derived from \eqref{sec:trajectory} 
From $\Phi(t,t_0)$ we can obtain not only the temporal evolution of the
solution $\X(t)$ but also of the distributions of ages of the mass in the compartments and in the entire system \cite{Metzler2018PNAS}.
The \CompartmentalSystems{} package provides all the functionality necessary to do these computations, which rely on a description of the time-dependent input vector $u(t)$ and the compartmental matrix $B(t)$, as well as initial age distributions for the compartments.
{todo @Markus, again, legend for notations is really necessary. E.g. matrix B(t) was never intorduced before, instead M(t) was used as a compartmental matrix. A person reading the paper will be very confused}
Furthermore, these computations can also be obtained for nonlinear systems of the form
\begin{equation} \label{eq:NANLS}
\dot{\X}(t) = \I(t, \X) + M(t, \X) \, \X(t), \qquad  \mathrm{with} \quad \X(t_0) = \X_0.
\end{equation}
by numerically solving \eqref{eq:NANLS} and plugging the solution $x(t, t_0,
x_0)$ back into it, which results in $\tilde{M}(t)=B(t, \X(t, t_0,\X_0))$ and
$\tilde{\I}(t)=\I(t,\X(t, t_0,x_0)$  i.e a linear system in the form
\eqref{eq:NALS}. 
{@Markus, please fix the next sentence - it is broken}
Therefore,  age and transit time
distributions can be obtained for nonlinear non-autonomous systemreimplements along the
specific trajectory. Detailed methods for the computation are provided in
\cite{Metzler2018PNAS}

%\section{Software description}
\section{The \python{} packages}
\label{sec:PythonPackages}
\subsection{\LAPM{}}
Linear Autonomous Pool Models (\LAPM{}) is a python 3 package for the study of autonomous compartmental systems at equilibrium such as those described in section \ref{sec:Equilibrium}. 
It implements the LinearAutonomousPoolModel class, with methods for the
symbolic and numerical solutions of the steady state content in the compartments, and the steady state release out of the compartments. For transit time, system age, and pool age, it provides symbolic and numerical computations of distribution densities, cumulative distribution functions, mean, standard deviation,           variance, higher order moments, and Laplace transforms. 

For the analysis of compartmental systems in analogy to absorbing Markov chains, \LAPM{} provides methods for the computation of the entropy rate per jump, the entropy rate per unit time, and path entropy. It provides the class DTMC (discrete-time Markov chains), with methods to compute the fundamental matrix, stationary distribution, and expected number of jumps of the Markov chain.

\subsection{\CompartmentalSystems{}}
This package deals with non-equilibrium trajectories of compartmental systems.
In particular, it provides the class smooth\_reservoir\_model to describe
symbolically the general class of non-autonomous nonlinear compartmental
dynamical systems of equation \eqref{eq:CompartmentalSystem}. It does not
require code for numerical computations or model simulations, but rather defines the underlying structure of the respective model. 
All fluxes or matrix entries are expected to be \sympy{} 
expressions. 

To obtain numerical results, the package
provides the class smooth\_model\_run, which is initialized with the initial
conditions of the system of equations, a set of parameter values, and a time
sequence. It computes the solution trajectory for the given initial conditions and
parameter values , finds the corresponding linear system with the same solution
following the strategy described in section \ref{sec:trajectory} computes the
state transition operator $\Phi(t, t_0)$ for these solution trajectories and 
provides methods to obtain time dependent densities with corresponding moments and quantiles for system age, compartment age, and transit time. 

An additional module provides functions to obtain initial age distributions
required for the computation of time-dependent age distributions. 
This module uses \LAPM{}.

\subsection{\ComputabilityGraphs{}}
This package was specifically developed for use in \texttt{bgc\_md2} but is also usable in separation.
It allows a declarative description of results (model properties) i.e. completely abstracting from the way they are obtained as e.g in a \texttt{Make} 
%\cite{
%Feldman, S. I. (April 1979). "Make --- A Program for Maintaining Computer Programs". Software: Practice and Experience. 9 (4): 255–265. CiteSeerX 10.1.1.39.7058. doi:10.1002/spe.4380090402. S2CID 33059412
%} 
\footnote{Which won the ACM software system award in 2002} file.  This
abstraction is the precondition for queries as in other declarative languages
like e.g. \texttt{SQL}, whose purpose is the comparison of data, as opposed to
the implementation of data base software.  Comparison of models w.r.t. their
predictions poses a similar challenge: The amount of e.g. Carbon or Nitrogen in
a compartmental system or their ages or transit times through it are (in
principle) measurable quantities. They do not change if the description of the
model is changed from a graph (pools and fluxes) to a matrix or a product of matrices.
Instead of forcing a user of the data base to use a fixed
description it is much more desirable to allow as many equivalent ones as
possible \emph{and} make the equivalence explicit by  well defined mappings i.e. \emph{functions}.

\ComputabilityGraphs{} facilitates exactly this.
It implements a class \texttt{CMTVS} which stands for 
{\bf C}onnected {\bf M} ulti {\bf T}yped {\bf V}ariable {\bf S}et. 
Instances consist of a set of variables with unique type (only one variable per type) and a set of type annotated functions that exclusively 
use these types in their signature (as arguments or return values) which we call \emph{computers} from now on.
This combination implicitly implements a declarative model description language to describe results of computations by requesting the type of the result. 

It also confines what we consider a model property, building block or
metric. The network of functions using these types as arguments or return values
enforces an unambiguous definition, which prevents the description of models to
become vague, e.g. the compartmental matrix is a function of the internal and outgoing fluxes and the ordering of the state variables.  
This rigorous description actually defines computability graphs that we exploit mainly in the following ways.
\begin{enumerate}
  \item
  \label{enum:computable}
  To compute which types of information (target results) are obtainable given the types of the provided variables. 
  Since a \texttt{CMTVS} instance knows the types of all its variables and the signatures of all
  available functions, it can iteratively add the result types of all applicable functions until no
  more result types can be reached. This is a forward graph search (\figref{fig:closure}).
    \begin{figure}[h]
      \includegraphics[width=\textwidth]{closure.pdf}
      \caption{Closure under computability} 

      The left hand side picture shows a first step in the computation of the
      computable types.  
      The dark green nodes on the left represent the types of the given variables.
      (in this case describing the YIBS model)
      We find a first applicable (i.e. all its arguments are
      given) function (grey node on the right), infer the result type from its
      type annotation and add it to the set of given types (new light green
      node).  Now we have more arguments and thus more possibly applicable
      functions.

      The right hand side plot shows the result of the recursive application of this
      procedure to a \texttt{CMTVS} instance.
      Without performing any actual computation we know which results we can
      compute (30 light green nodes) from the 11 provided variables (dark
      green), some of them in different ways via intermediate results
      (as explained in  \figref{fig:dep_graph} below).  This
      information is used to automatically add methods to an \texttt{CMTVS}
      instance, so that interactive python environments suggest computable
      results to the user. In this case 30 new methods appear automatically, including very complex results like the pool specific transient 
      mean age solution for the vegetation carbon sub system witch appears as \texttt{get_NumericVegetationCarbonMeanAgeSolutionArray()}
      It is also the basis for queries, e.g. "for which
      models can we compute the mean transit time through the 
      the vegetation subsystem?"  Note that the applicable functions (in this
      example 45 represented by the grey nodes) can also be used
      independently of the \texttt{CMTVS} class, explicitly by the user. 
      A lack of the automatic combination would however make it much more difficult to
      guide a user through the often purely technical conversions to the
      targeted result and massively increase the amount of necessary
      handwritten documentation.  In this sense \ComputabilityGraphs{} can be
      used as computable documentation or a much simpler API.  
      \label{fig:closure}
    \end{figure}
  \item
  \label{enum:deptree}
  To find out what type of information is 
  \emph{missing} to obtain a 
    target variable (backward search).  
  \figref{fig:dep_graph} shows the bipartite dependency tree for a quite complex numerical result,
  printed by the \texttt{CMTVS} instance. 
  The red node T35
  at the bottom represents the target that we want to
  compute: \\
  \texttt{NumericVegetationCarbonMeanBackwardTransitTimeSolution}. 
  The green nodes are the building blocks that we have provided in the
  model description so far.
  The light red nodes show that are not provided but have to be 
  computed via the function represented by the grey nodes. 
  This is possible if all their (ultimate recursive) dependencies
  are green nodes. 
  In this example this is the case except for nodes  T70 and T38 ,
  \texttt{StartConditionMaker} \footnote{Variables of this type are actually a functions that can compute a set of consistent start conditions (mass, age density, mean age), which we note here to show that the concept of type is not confined to a traditional data types.  
  } and 
  \texttt{VegetationCarbonStateVariableTuple} 
  respectively. 
  \item
  \label{enum:compute}
  To actually  compute the result of a targeted type.
  If a result type is in the computable set determined by \enumref{enum:computable} then 
  the search tree created under \enumref{enum:deptree} can be traversed in reverse, starting
  at the given nodes and ending in the final result.
\begin{figure}[h]
  \includegraphics[width=\textwidth]{dep_graph.pdf}
  \caption{ Dependency graph}
  {
    \small
    \begin{tabular}{l l}
	T1 	 & 	InFluxesBySymbol\\
	T5 	 & 	VegetationCarbonSmoothModelRun\\
	T8 	 & 	CompartmentalMatrix\\
	T9 	 & 	SmoothModelRun\\
	T12 	 & 	StateVariableTuple\\
	T13 	 & 	NumericParameterization\\
	T18 	 & 	InputTuple\\
	T23 	 & 	NumericSimulationTimes\\
	T26 	 & 	OutFluxesBySymbol\\
	T35 	 & 	NumericVegetationCarbonMeanBackwardTransitTimeSolution\\
	T38 	 & 	VegetationCarbonStateVariableTuple\\
	T39 	 & 	SmoothReservoirModel\\
	T43 	 & 	TimeSymbol\\
	T49 	 & 	NumericVegetationCarbonStartMeanAgeTuple\\
	T54 	 & 	NumericStartValueArray\\
	T55 	 & 	NumericParameterizedVegetationCarbonSmoothReservoirModel\\
	T66 	 & 	NumericParameterizedSmoothReservoirModel\\
	T68 	 & 	InternalFluxesBySymbol\\
	T70 	 & 	StartConditionMaker
\end{tabular}
  }  
  \label{fig:dep_graph}
\end{figure}  
\end{enumerate} 
Using \enumref{enum:computable} and \enumref{enum:compute} together a
\texttt{CMTVS} can add get methods for computable results dynamically.  This
is very useful for the user interface in interactive python sessions,
including Jupyter notebooks since on pressing the tab key, the python
interpreter suggest available method calls as autocompletion options for any
object followed by a "." For an \texttt{CMTVS} object these methods become
more numerous automatically as more and more information is added to it.
Apart from the user interface \enumref{enum:computable} is necessary for queries.
If we have a (large) set of different \texttt{CMTVS} objects and want to
compare them with respect to a certain property, we must first find out for
which of them this property is actually computable.  A welcome side effect
of the \text{CMTVS} model description language is the possibility to include
models about which we know very little. In a traditional database these
record would most likely be incomplete, whereas here we just get offered
fewer computable results.  
This is especially useful for the creation of new
models, in the process of which one naturally starts with a small set of
variables that is gradually extended.
Typically one of the first steps is a dictionary of symbolic flux equations, which already allows the visualization
of the compartmental graph or flow diagram, symbolic matrices and vectors and is extremely useful for debugging.



\subsection{The biogeochemical model database \texttt{bgc\_md2}} This is the
central package and can be seen as a front end to \CompartmentalSystems{} and
\LAPM{} facilitated by \ComputabilityGraphs{}.  For the following discussion it is
important to note that \texttt{bgc\_md2} is a library, rather than a framework,
since there is no `inversion of control', i.e. \texttt{bgc\_md2} as well as all
the other packages are called from normal python code, rather than
\texttt{bgc\_md2} calling user code in a restricted execution environment as a
framework would do. \todo { @Markus, DSL (sentence below) was explained in the abstract, but not in the main text. PLease add a sentence here to introduce this concept.} Although internally \ComputabilityGraphs{} acts like an
extended  compiler  (not only testing type consistency of function calls but
actually suggesting them) for a DSL tailored to the domain of compartmental
models, it is technically represented by \texttt{bgc\_md2}'s API (consisting of
the provided types, functions, and the dynamically created methods of the
\texttt{CMTVS} objects for computable properties).  This makes it much more
flexible to use than a framework. We can use the full tool set of the
\emph{python} ecosystem to create instances of the model building blocks and to
postprocess results afterwards.  While any comparison requires standards i.e.
rigorously defined properties which can be formulated for all models and
therefore naturally suppresses idiosyncrasies of models, it can be very helpful
to use these idiosyncrasies for the creation of a model.  This is possible
since we can use absolutely unrestricted python code to create the model
building blocks.  Imagine for example a model that includes pools in many soil
layers with similar connecting fluxes. A model description code using
\texttt{bgc\_md2} could use loops to express this.  This also applies to
postprocessing. Some of the functionality in \LAPM{} and \CompartmentalSystems{} is
not generally applicable to \emph{all} models in \texttt{bgc\_md2}'s collection
and therefore cannot (and should not) 
be offered by the `computers'. 
This very deliberate restriction is however no problem since 
nothing prevents a user to apply them (or any other bit of code) to the results of a
\texttt{bgc\_md2} computations.  
Armed with this flexibility we know that our
DSL does not have to provide everything and can restrict it to sensible categories
for comparisons.
%\subsubsection{Provided Types and Computers}
In particular \texttt{bgc\_md2} provides:
  \begin{enumerate}
    \item
      Types defining {\bf building blocks and comparable diagnostic results} 
      specifically tailored to compartmental models, 
      e.g.\ \texttt{CompartmentalMatrix}, \texttt{InternalFluxesBySymbol},
      \texttt{NumericVegetationCarbonMeanBackwardTransitTimeSolution} \dots  
      \footnote{A complete list can be produced by a single command}
  %and is added to \appendixref{appendix:listing}
      specifically including the symbolic expressions for pool contents, in- out- and internal-fluxes for
      subsystems, single pools or the entire model, as well as their numerical counterparts obtained from 
      by solving the parameterized IVPs and the most general diagnostics e.g. transient mean-age and transit time solutions.
      Since many of these types are based on a symbolic
      mathematical representation, using \sympy{}, many symbolic results are computable 
      without the need to know parameters or data to run the
      models.  Using the underlying graph representation as sets
      of pools and fluxes, we can reorder pools and thereby
      automatically transform the compartmental matrices, group
      them into different subsets (e.g. vegetation, soil, litter,
      carbon, nitrogen \dots ), substitute pool names, get
      mathematical expression for cumulative fluxes e.g. from
      vegetation to soil or simply plot the graphs (\figref{fig:subsystems}).
      Using this symbolic transformations we can compare models that might
      have looked very different initially, simply due to
      arbitrary choices of pool names or their even more arbitrary
      order in which they appear. 
      \begin{figure}[h]
      \includegraphics[width=\textwidth]{"kv_visit2_veg_soil_decomposition.pdf}
      \caption{Automatic decomposition into subsystems} 
      Assuming that a compartmental system has been defined symbolically e.g. by providing expressions for input,  output and internal fluxes different flow diagrams can be created automatically. 
      The additional information for the decomposition into subsystems just consists of a set of pool names for each subsystem.
      (Here vegetation and soil part for the visit model reconstruction).
      For special subsystems like vegetation or soil that frequently occur in Carbon cycle models a declaration of a set of e.g. soil pools has far reaching consequences. 
        Apart from the graph shown here, such a statement immediately makes
        several new diagnostics available, including matrix descriptions for
        the vegetation subsystem as in \cite{Ceballos2018Biogeosciences} or
        their soil equivalent, cumulative fluxes between the two subsystems as
        well as transient age and transit time distributions w.r.t. the
        vegetation or soil subsystems.
        
      \label{fig:subsystems}
      \end{figure}
      
    \item
      A set of type annotated functions (from now on called computers) operating on those types,  which combined
      by \ComputabilityGraphs{} form a much simplified 
      interface to \emph{many  algorithms} in \texttt{CompartmentalSystems} and \LAPM{} to compute diagnostic variables
      for \emph{many models} in \texttt{bgc\_md2}.
      %A very simple example is depicted in \figref{fig:subsystems}.
    \item
    $30+$ vegetation, soil or ecosystem models for carbon and nitrogen cycling
      as reusable python modules using the building blocks in a flexible way. 
  \end{enumerate}

\subsubsection{Intentionally Missing Computers}
\todo {Comment from Kostia: I'm not sure this paragraph is neccessary the way it is framed. There is already an idea in the previous paragraph that non-universal things should not be in 'computers'.
I don't think we should dwell on this too much. Unless it can be re-phrazed in a positive manner, e.g. \"Although certain features are not implemented in computers, they are demonstraited in the example notebooks...\" In this case it can be merged with the following Section 4}
It is also interesting to note what \texttt{bgc\_md2} intentionally does not implement:
A trivial example is that the computation of equilibria is missing from \texttt{bgc\_md2}'s set of strictly typed functions. 
While we can easily compute the uniquely defined fixed point of a linear
autonomous system, and indeed provide functions to do so in \LAPM{}, we have to
resist the temptation to make them available via \texttt{bgc\_md2}'s set of
types and computers for \emph{all} compartmental systems. 
This restriction also pertains to equilibrium start age distributions. 
If they simulation is to be started from an equilibrium this is reflected not only by the correct equilibrium
but also a specific age distribution.  
At the moment neither equilibria nor the start age distributions 
are inferred automatically, but have to be provided per model by 
user code to clearly define responsibilities (although, thanks to \LAPM{}, the user code amounts to only three lines).
While we could in the future provide special types for liner matrices or
fluxes, constant rates, along with their appropriate constructors implementing
automatically testable criteria for accidental misdeclaration, implement
computers for them and thus formalize comparisons between linear models
this has not been done yet and such results are not suggested automatically by \texttt{bgc\_md2}.
Users can however use the many functions in \LAPM{} or \CompartmentalSystems{} directly.

We also avoid `computers' depending directly or indirectly on ambiguously defined matrix factorizations, 
as we mentioned above and in explained in detail in \appendixref{appendix:MatrixDerivation}. 
These computations can still be performed aided by  \LAPM{} or \CompartmentalSystems{} on intermediate results
provides by \texttt{bgc\_md2} computers but also not automatically, since results would be ambiguous (in contradiction to the mathematical definition of a function which is our criteria  for admission). 


%%%%%%%%%%%%%%%%%%%%%%%%%%%%%%%%%%%%%%%%%%%%%%%%%%%%%%%%%%%%%%%%%%%%%%
\section{Example Applications}  
\label{sec:ExampleApplications}
It is impossible to fully exhibit the potential of this approach without
examples.
To this end we created some illustrative \texttt{Jupyter} notebooks that are
available via binder online without the need to install the packages. 
This paragraph contains only pointers to those much more elaborate notebooks
and some example plots from them.
The examples demonstrate how \texttt{bgc_md2}:
\begin{enumerate}
\item 
  Simplifies and unifies the creation of a new model from scratch while using the
  symbolic and graphic diagnostic capabilities to inspect it while we build it
  and use 
  \ComputabilityGraphs{} to point out what missing information we have to
  provide to make desired results computable.
  This is demonstrated in
    \href{https://github.com/MPIBGC-TEE/bgc_md2/blob/binder/binder_notebooks/illustrativeExamples/createModel.py}{binder_notebooks/illustrativeExamples/createModel.py}
    
\item 
  aided by \ComputabilityGraphs{} 's ability to compute which properties are computable
  allows to query the collection of models that are already part of
  \texttt{bgc\_md2}.
  This is demonstrated in
    \href{https://github.com/MPIBGC-TEE/bgc_md2/blob/binder/binder_notebooks/illustrativeExamples/databaseAspects.py}{binder_notebooks/illustrativeExamples/databaseAspects.py}
%\item 
%  How a model that is already part of the database can be
%  inspected w.r.t complex diagnostics like the transient mean transit time that
%  would take years to implement in stand alone model code but are now available
%  for all models with sufficient information.
%  This is demonstrated in
%  %\url{}
\item 
  \label{enum:trendy}
  Several Models for which numeric parameter values and driver data are
  available can be compared w.r.t. abstract properties.
  We chose the mean age and transit time of the vegetation and soil subsystems
  since they are defined for all models while the concrete pools of the models
  differ and not only have different names but also different meaning.
  \todo{@Markus Fig. 6: we talked about a problem of period in transit time computation.
  Also, I would point out the magnitude: soil transit time is only 2 times higher than vegetation, while I would expect at least 10 times higher. This is probably due to poor parametrization.
  Fig. 7: Ensabmle mean is not visible in the image}
  (\figref{fig:btt_4models}, \figref{fig:stock_mean} )
\end{enumerate}
We will look closer at  
  \enumref{enum:trendy}. It is interesting to see how only the same 10 variables 
  as shown in \figref{fig:closure} 
      (
          T18: InFluxesBySymbol, 
        	T20: InternalFluxesBySymbol, 
        	T31: NumericParameterization, 
        	T35: NumericSimulationTimes, 
        	T46: OutFluxesBySymbol, 
        	T56: SoilCarbonStateVariableTuple, 
        	T57: StartConditionMaker, 
        	T59: StateVariableTuple, 
        	T60: TimeSymbol, 
        	T69: VegetationCarbonStateVariableTuple
      )
  are actually given,
  and how it is possible to compute such a relative complex result
  like the mean backward transit time for the vegetation part automatically
  \figref{fig:dep_graph_2}.
%%%%%%%%%%%%%%%%%%%%%%%%%%%%%%
\begin{figure}[ht]
	\includegraphics[width=\columnwidth]{bak_dep_graph_2.pdf}
    \caption{
      Possible ways to compute the backward transit time for the vegetation
      subsystem, from the few variables given in the description of the YIBS
      model (\figref{fig:closure} in dark green).
    }
    \begin{itemize}
      \item
      Many \emph{labels} (e.g. T59)  appear more than once in the graph, while of
        course a node itself cannot.  The reason is that the nodes in this
        search tree are actually tuples of which we only show the first part in
        the label. The second part, \emph{which is invisible here}, is a set of
        types that, in the branch of the node, have already been used and are not
        allowed anymore.  This is a standard technique to transform search
        \emph{graphs} (with possible circular dependencies) into search
        \emph{trees} which can be traversed much faster.
      \item
        There are actually only nine variables necessary (\figref{fig:closure} dark green)
        (
          T18: InFluxesBySymbol, 
        	T20: InternalFluxesBySymbol, 
        	T31: NumericParameterization, 
        	T35: NumericSimulationTimes, 
        	T46: OutFluxesBySymbol, 
        	T57: StartConditionMaker, 
        	T59: StateVariableTuple, 
        	T60: TimeSymbol, 
        	T69: VegetationCarbonStateVariableTuple
       ) to compute the result.
      \item
        There are actually different ways to compute the result. Some of the red nodes (e.g. T41)         could be computed by different functions (here f44 and f59)  which have different arguments which are compute by different functions \dots so the tree branches out. 
        At the moment the algorithm chooses the first path but we could use this feature to
        remove hidden duplication in the model description (remove some of the given variables since they are computable) and 
        if it is unavoidable to automatically test its consistency. 
    \end{itemize}
  \label{fig:dep_graph_2}
\end{figure}
%%%%%%%%%%%%%%%%%%%%%%%%%%%%%%
\todo{@Holger, Kostia: \\
We should try to publish the following in PNAS as an unexpected short Corollary to \cite{Metzler2018PNAS} called `Inverse Pool Algebra', because it is generalizable to any subsystem.
The actual computation can be made more elegant: 
Similar to the extension of the original IVP to the age moment system, we can add more equations for the subsystem age moments and solve them simultaneously without first solving the original system and subsequently substituting the solutions as described below.
One special case is the computation of "pool" ages (age w.r.t. entry into a pool not the whole system) 
 with IVPs (instead of our more elaborate (unpublished) approach with the bins that also works for non-well mixed). 
 Kostia asked me to do it (and patiently waited until the plots looked reasonable) and after only several refactorings I realized that this is the "Pool Algebra backwards" 
 Its actually quite surprising that we can compute the age with respect to a set of pools in the middle with an IVP since the whole well mixed approach has no explicit 
 concept of crossing a pool boundary for a single particle (only implicitly by the Markov Chain Interpretation...)     
 The connection is the start condition (does not have to be equilibrium based, pinup or empty would also do) and the Input Age of zero for the flux into the subsystems, Crazy sh...}\\
The short version of the mathematical description is as follows:
\begin{enumerate} 
  \item 
    Assemble the ODEs for the whole system, from the flux
    equations for in- out- and internal fluxes that are given in the model
    descriptions.  
  \item 
    Apply the \texttt{StartConditionMaker} a function that
    yields a consistent tuple of contents, mean ages, and age distributions
    for all pools to the ODE.  (In this case based on the assumption of the
    trendy9 S2 experiment of starting in equilibrium, which can only apply to
    a frozen system within which all time dependent functions have been
    substituted by there value at $t=0$.  Other ways to compute
    start conditions is to assume empty pools, with age and mean age $0$ or the
    result of a `spin up' from this situation for given or assumed functions of
    \emph{past} times.) 
  \item 
    Solve the IVP for the whole system.
    \label{enum:solveIVP} 
  \item 
    \label{enum:symVeg} Create the ODEs for the
    vegetation subsystem: Extract the sub graph for the vegetation pools, find
    all fluxes into it and out of it.  Use the symbolic representation of the
    flux functions to add up the incoming fluxes to a Vegetation Input.
    Compute the CompartmentalMatrix for the vegetation subsystem (which will
    also depend on pools, that are NOT part of this subsystem.) 
  \item
        \label{enum:numVeg} Substitute the solutions for the pool contents
        computed under \enumref{enum:solveIVP}.  into the Influx and matrix
        expressions computed under \enumref{enum:symVeg}

  \item 
      Apply the (same) \texttt{StartConditionMaker} to the vegetation
      subsystem.  Together with \enumref{enum:symVeg} this forms a new
      vegetation IVP.
      
  \item 
    Extend the vegetation IVP by the first age moment (mean) equations.
  \item 
    \label{enum:meanAgeVeg} Solve the mean age vegetation IVP. 
    % (including the matrix ODE for $\Phi$ from \eqnref{eqn:Phi}, which
    While this (unnecessarily) reproduces the solution for the vegetation
    pool contents as function of time, the mean age solution differs because
    influx into the vegetation system (that actually comes from other pools) now
    has age 0.  \item \label{enum:numOutVeg} Substitute the solution computed under
    \enumref{enum:solveIVP} into the fluxes out of the vegetation pools (w.r.t.
    the whole system these are either outfluxes, like autotrophic respiration,
    or internal fluxes e.g. fluxes to soil or litter pools).  This makes
    these fluxes functions of time.  \item The mean backward transit time is
    the mean age of the material in the vegetation outfluxes and can be
    computed from the results of \enumref{enum:meanAgeVeg} and
    \enumref{enum:numOutVeg}.  
\end{enumerate} 


We could have implemented a separate function to do all this (which we actually did in the beginning),
but we broke it down in to much smaller functions that only compute one step and now \ComputabilityGraphs{} 
reconstructs the whole computation from these smaller bits.  
This has several consequences:
\begin{enumerate}
  \item
    The necessary user code is one line long:\\
    \begin{minted}{python}
      mvs.get_NumericVegetationCarbonMeanBackwardTransitTimeSolution() 
    \end{minted}
    Where mvs is the \texttt{CMTVS} instance imported from the data base for the
    respective model.  The result of this is an array that can be plotted over
    time as we show in \figref{fig:btt_4models}. 
    The computation for the soil
    part is nearly identical except for a different subset of nodes that leads to a
    different subgraph.
  \item 
    The code is \emph{declarative} i.e only describes the result not the computation.
    In particular the result would have been obtained by the same line of code if the model had been described
    using other building blocks, (\texttt{CompartmentalMatrix} and \texttt{InputTuple})
    because \ComputabilityGraphs{} would have looked for, found and applied the additional 
    conversion functions automatically.
  \item
    Because the code is declarative we could even have iterated over all the models (implemented as python modules)
    in the `data base' to first find all for which we \emph{can} compute it (somehow) and then do it. 
    (Although in this case the choice of model was more constrained by the availability of common driver data)

\end{enumerate}


\begin{figure}[t] \includegraphics[width=\columnwidth]{test_veg_soil.pdf}
  \caption{
    Figure above: Comparing the transient backward transit time through
    subsystems, across different models.
  } 
  \label{fig:btt_4models} 
\end{figure}  

\begin{figure}[t] \includegraphics[width=\columnwidth]{test_stock_mean.pdf}
  \caption{
    Cumulative change of Carbon %\(C\) 
    stock since pre-industrial conditions predicted by 5 models from the \texttt{bgc\_md2} database. Grey area indicates $\pm$ one standard deviation around the ensamble mean.
    Top panel: total C stock; middle panel: vegetation C; bottom panel: soil C.  \texttt{bgc\_md2} can automatically compute diagnostics for sub-systems: the user just needs to specify which pools
    belong to which subsystem.
  }
  \label{fig:stock_mean} 
\end{figure}  


\section{Conclusions} \label{sec:ConclusionsAndOutlook}
We implemented a
practically usable tool for the creation, collection, inspection and comparison
of compartmental models and demonstrated how it could be applied quickly to the
practical problem, of comparing several models with respect to a numerical
result complex enough to make its implementation on a per model basis
prohibitively expensive.  We created examples to teach new users how to use the
existing capabilities, how to query the collection and how to add models to it.
These immediate practical benefits are sufficient for many simple applications and
are to our knowledge unprecedented.  In the process some lessons have been
learned, that could be valuable to anybody undertaking a similar project or
interested in extending our work. \todo{@Markus I cannot understand the following sentence at all. Please consider refrasing:} An important one is the fertile feedback of
the self imposed restriction to properties that are either results or arguments
of computations on the clarity of the description of features.  Another one is
the implementation of all the tools as libraries rather than a framework, which
allows the use of the provided tools in situations that we do not anticipate yet
and allows us to be strict in the data base part without creating a bottleneck.
We have also collected some more general observations: Creating collections of
many models and many metrics consistently applicable to all of them is hard,
finding well defined ways to allow users to contribute even harder, documenting
them comprehensively harder still.  The necessity for rigorous definition,
absolute clarity and avoidance of duplication becomes more and more important
as the collection of models grows.  While we have reached a point where the
contribution of new models is easy enough to be done by the users of the
software, implementation of new features will likely
require the permanent attention of a technical project lead for 
the lifetime of the project, since occasional refactoring is likely to be necessary
to keep the code base maintainable.
\subsection{Possible Extensions and Improvements}
\subsubsection{More Models}
The easiest way to extend the data base is to contribute new models.
We have created an example notebook as an interactive tutorial.
Since very few building blocks are necessary, adding a new model is actually very simple.
The symbolic part takes usually much less than a day, 
and adding parameters much less time than finding them in the literature.
Since \texttt{CMTVS} instances are normal python objects and have update and remove methods, models can use other models as source and extend them dynamically.
The already stored models descriptions are \python{} modules and can import others or be imported by them.
When you are happy with your code a pull request will put your model into the default location 
and make it subject to the tests. 
\subsubsection{More Real Parametrizations and Driver data}
Many of the model descriptions contain a variable of type \texttt{NumericParameterization} 
which contains dictionaries for parameter values and functions, mapping the symbolic description to a runnable model.
These \texttt{NumericParameterization} instances are usually extremely small example data sets, whose purpose is to give users some arguments to test the numerical results with.
This is intended. \texttt{bgc\_md2} is not a place to store data.
However we would like to apply models to real data. Usually via model and data set specific code that makes available data accessible to our models.
It is surprisingly tedious \todo{Big THANKS to Veronika, for doing it for many models} and for some models impossible to find scientifically meaningful parameters in the literature.
In the case of earth system models we would need such a set of parameters and driver functions for each spatial pixel.
Even for model intercomparison projects such as TRENDY \cite{sitch_recent_2015} these parameters are usually not publicly available and it is 
often not possible to estimate them accurately even from synthetic model output. 
\footnote{
  Usually due to identifiability issues. 
  30 or more parameters can hardly be expected to be estimable from much fewer data streams
}
We have created a small trendy9 specific package for gluing the data to \texttt{bgc\_md}'s models, with parameter estimation code for four models.
It would be very beneficial to have more gluing code for more data sets available...


\subsubsection{New Metrics and Building Blocks}
Adding `computers' is technically very simple. It will make you think though.
In the `language' of \texttt{bgc\_md2} anything worth saying about a compartmental model is the result of a computation, or an argument that influences other computations or most likely both.
Adding a new computer is the simplest case, since one has only to make sure that for \emph{any} arguments that fit the type signature a unique result can be computed. If new argument types or new result types have to be introduced it is usually worth asking if these arguments could be computed from something simpler. The reiteration of this process can lead to astonishing results.
At some point we wanted to add information about the vegetation subsystems of some ecosystem models. 
We needed the related subsystem compartmental matrix as in \cite{Ceballos2018Biogeosciences}. 
Adding it as a model property would have been very error prone, adding a projection operation on the matrix would have been possible but far more complicated
than the graph description that is now implemented and allows to specify a
vegetation state variable tuple from which not only the vegetation
compartmental matrix can be computed but any other vegetation related result too. 
Adding a new word to the vocabulary is a very powerful extension.

\subsubsection{Algebraic Types}
The sets of `computers' and their argument and result types is by no means comprehensive.
In fact there are many results that could be implemented quite easily. 
E.g. we implemented the two types \texttt{VegetationCarbonStateVariableTuple} and 
\texttt{SoilCarbonStateVariableTuple}. It would have been easy to created something similar for e.g. Nitrogen. However the number of types would quickly become large since their
are also related types that would have to be duplicated, i.e. \texttt{NumericVegetationCarbonSolutionArray} or \texttt{NumericVegetationCarbonMeanBackwardTransitTimeSolution}...
To combat this `combinatoric explosion' it would be more elegant to express these relationships between types by a concept called
algebraic data types which is used e.g. in  \texttt{Haskell} but also available in \python{} 's type hint syntax but not yet implemented in \ComputabilityGraphs{}.
The concept is actually very simple as e.g. a type that uses another type as parameter like a list of integers \texttt{List[int]}. Algebraic types would give us a bigger namespace 
for less awkward type names and also avoid some boilerplate code for the `computers'.
\footnote{R does allow polymorphism in all arguments by means of \texttt{S4} classes, but does not allow algebraic types. So in SoilR the combinatoric explosion also lead to longer and longer type names.
}
Another related extension of \ComputabilityGraphs{} is to allow `computers' that return tuples. 
This could be used to minimize the number of recursive computer application for sets of desired results.   

\subsubsection{Automatic Code Generation, Other Languages and Integration into Earth System Models}
\texttt{bgc\_md2} uses symbolic math to describe the model equations. It then
uses \sympy{}'s facilities to first translate these expressions into regular python code and execute it in a special environment that can be influenced.
\sympy{}
provides a printer module that allows the translation of \sympy{} 
expressions to many different programming languages including but not limited to C, Fortran, Java, R and Julia.
So that models could also be translated to these languages automatically. 
A possible application of this ability would be to automatically compile different land-carbon models from \texttt{bgc\_md2}'s collection into an earth system model without the need to reimplement them in the targeted language. 
Remarkably \ComputabilityGraphs could also easily be extended to use collections of `computers' in other languages. 
It acts much like a compiler and could use the graphs to automatically create code calling the `computers' recursively in the right order. 
\footnote{At the moment this is not necessary since the computations traversing the graph in reverse are all performed immediately in python with with a growing intermediate result dictionary}  
This is extremely useful for the next possible extension.

\subsubsection{Benchmarking Testbed and Stepping Stone for the Implementations of Diagnostics in Other Projects}
Our packages have hundreds of unit test cases. If the integration of transit time computations into an existing earth system model was desired, these test could be adapted
much faster than created from scratch. For nonlinear and nonautonomous systems, where analytical tests are hard to come by, \texttt{bgc\_md} can be used as a benchmark for any desired model (which could be implemented much more quickly there and then used as a reference in a set of tests for the new implementation.

\subsubsection{Documentation, Error Messages}
Our software tools are written in \python{}. 
This has advantages:
\begin{itemize}
  \item
  The availability of hundreds of libraries for pre-  or postprocessing of model building blocks or results.
  \item
  The flexibility of a very permissive dynamically typed language which allows for rapid prototyping and interactive use of the tools we created.
\end{itemize}
but also disadvantages: The same permissiveness that allows for rapid prototyping and interactive use can also make the detection of unintended misuse harder.
In strictly typed languages like Haskell many unintended usecases are already detected at compile time and reported as errors.
This 'fail early and hard' philosophy, facilitates the creation of 'fool proof'
'software', whereas interactive environments, notably  R and python follow
almost the opposite approach.  They try their best to make sense of anything
thrown at them which can delay the detection of an error and thereby remove the effects far from the cause.
Complex libraries like \texttt{bgc\_md2} which sits on top of \CompartmentalSystems{} which sits on top of \LAPM{} which sits on top of \texttt{numpy}, \texttt{SciPy} and \sympy{}
need extra care to filter out user input that could cause trouble further down the call stack. 
\footnote{E.g. in some corner cases the underlying libraries might treat a matrix of dimensions n,1 differently than an array of dimension (n,), while users usually do not anticipate this.} This becomes more important the less obvious the computation in question is. 
While an example makes the user aware of their responsibility for correct adaptation, the same user will expect the API of a `black box' to be 'fool proof'. 
This is an ongoing task: While \LAPM{} and \CompartmentalSystems{} are pretty well documented 
\todo{Big Thanks to Holger!!!} 
\texttt{bgc\_md} and \ComputabilityGraphs{} are not yet and none of the packages is `fool proof'. 

\subsubsection{Polished UI, Web Interface}
We implemented some widgets to use in \texttt{Jupyter} notebooks (e.g. to show a subset
of the models in an interactive table or to explore the computability graphs).
These widgets are absolutely basic  proofs of concept, mainly intended to icite a skilled web developer 
(or enthusiastic student in need of a project) to transform them into something much
more useful (a click on an edge in a computability graph could generate and print code for this path) and shiny.
This is especially true for the graph visualizations. 
Plotting (mathematical) graphs is a surprisingly hard problem and
standard plotting libraries like \texttt{matplotlib} or
\texttt{igrap} support only basic graph layout algorithms.  
Widgets or web applications using \texttt{javascript} would allow to use more specialized libraries like \texttt{Cytoscape.js} 
and make the interactive exploration or publications- ready printing of the compartmental model (as well as the computability graphs) 
much more beautiful. 
It would also be relatively easy to host \texttt{bgc\_md2} based on a server, preferably with access to many datasets.
A \texttt{Jupyter} server would be the most flexible option for scientific users, 
but specialized and less interactive websites for model or result presentations could 
easily be build using the existing example widgets as a starting points. 






%\subsection{Installation, documentation and demonstrations}
%The four packages are managed under version control in GitHub, and are publicly available for their evaluation, use, and further modification. 
%
%The four packages can be installed following the instructions provided for the
%installation of \texttt{bgc\_md2}, which depends on \LAPM{} and
%\CompartmentalSystems{} and \ComputabilityGraphs{}. 
%Several installation scripts exist that installs all packages at once. 
%the standard one using Conda. 
%All packages have extensive testsuites which are triggered by every commit but
%can also be run manually e.g. after installation which we strongly recommend.
%The test suites are  stored in the folder \texttt{tests} under the root of the repo. If they run successfully, then all the functionality of the four packages is readily accesible. 
%
%Detailed instructions of the installation procedure can be found in the `README.md' file of the \texttt{bgc\_md2} repository. 
%
%We use standard python docstrings to document modules, functions, classes and methods which are
%available in interactive python sessions. 
%We also compile them using Sphynx, a documentation generator for python code,
%into a hypertext documentation which is served by GitHub. It's automatically updated when changes to the documentation are pushed to the master branch of the repositories. 
%
%In addition to this documentation, we provide Jupyter notebooks for each package, with demonstrations of specific aspects of the available functionality.
%Each package has a 'notebooks' folder where the Jupyter notebooks are stored. However, they are stored as source.py files and not in the original.ipynb format. 
%The reason for this is that to maintain the notebooks under version control and avoid conflicts among different versions on different machines, we store the notebooks as regular python files. To convert these files to Jupyter notebooks, we recommend to use Jupytext, which does the translation with a simple `convert' command. 
%
%\section{Illustrative example}
%We provide an example notebook in
%https://github.com/MPIBGC-TEE/bgc_md2/tree/master/notebooks/illustrative_example.
%It shows the interplay of all four packages on the collection of predefined
%models in \texttt{bgc\_md2} and ways to extend this collection by defining a new model
%that can be compared with respect to diagnostic variables that are computed
%using \LAPM{} and \CompartmentalSystems{}.  The best way to use the example is to
%install the packages and explore the example interactively.
%\section{Impact}

%%% End of body of article

%%%%%%%%%%%%%%%%%%%%%%%%%%%%%%%%%%%%%%%%%%%%%%%
%% Optional Appendices go here
%
% The \appendix command resets counters and redefines section heads
%
% After typing \appendix
%
%\section{Here Is Appendix Title}
% will show
% A: Here Is Appendix Title

%%%%%%%%%%%%%%%%%%%%%%%%%%%%%%%%%%%%%%%%%%%%%%%%%%%%%%%%%%%%%%%%%%%%%%%%%%%%%%%%%%%%%%%%%%%%%%%%%%
\appendix
%\section{Technical details and listings}
%\label{appendix:listing}
%
%\begin{lstlisting}[language=Python]
%# We then look at all types that could possibly computed and find one that looks as if it
%# sounds like a solution.
%import ComputabilityGraphs.helpers as cgh
%computers=module_computers(bgc_md2.resolve.computers)
%types=cgh.all_mvars(computers)
%types
%
%frozenset({CompartmentalSystems.smooth_model_run.SmoothModelRun,
%           CompartmentalSystems.smooth_reservoir_model.SmoothReservoirModel,
%           bgc_md2.resolve.mvars.AggregatedSoil2VegetationCarbonFlux,
%           bgc_md2.resolve.mvars.AggregatedSoilCarbon,
%           bgc_md2.resolve.mvars.AggregatedSoilCarbonOutFlux,
%           bgc_md2.resolve.mvars.AggregatedVegetation2SoilCarbonFlux,
%           bgc_md2.resolve.mvars.AggregatedVegetationCarbon,
%           bgc_md2.resolve.mvars.AggregatedVegetationCarbonInFlux,
%           bgc_md2.resolve.mvars.AggregatedVegetationCarbonOutFlux,
%           bgc_md2.resolve.mvars.CarbonCompartmentalMatrix,
%           bgc_md2.resolve.mvars.CarbonInFluxesBySymbol,
%           bgc_md2.resolve.mvars.CarbonInputPartitioningTuple,
%           bgc_md2.resolve.mvars.CarbonInputScalar,
%           bgc_md2.resolve.mvars.CarbonInputTuple,
%           bgc_md2.resolve.mvars.CarbonInternalFluxesBySymbol,
%           bgc_md2.resolve.mvars.CarbonOutFluxesBySymbol,
%           bgc_md2.resolve.mvars.CarbonStateVariableTuple,
%           bgc_md2.resolve.mvars.CompartmentalMatrix,
%           bgc_md2.resolve.mvars.InFluxesBySymbol,
%           bgc_md2.resolve.mvars.InputTuple,
%           bgc_md2.resolve.mvars.InternalFluxesBySymbol,
%           bgc_md2.resolve.mvars.NitrogenCompartmentalMatrix,
%           bgc_md2.resolve.mvars.NitrogenInFluxesBySymbol,
%           bgc_md2.resolve.mvars.NitrogenInputTuple,
%           bgc_md2.resolve.mvars.NitrogenInternalFluxesBySymbol,
%           bgc_md2.resolve.mvars.NitrogenOutFluxesBySymbol,
%           bgc_md2.resolve.mvars.NitrogenStateVariableTuple,
%           bgc_md2.resolve.mvars.NumericCompartmentalMatrixFunc,
%           bgc_md2.resolve.mvars.NumericCompartmentalMatrixSolutionTuple,
%           bgc_md2.resolve.mvars.NumericMeanAgeSolutionArray,
%           bgc_md2.resolve.mvars.NumericMeanBackwardTransitTimeSolution,
%           bgc_md2.resolve.mvars.NumericParameterization,
%           bgc_md2.resolve.mvars.NumericParameterizedSmoothReservoirModel,
%           bgc_md2.resolve.mvars.NumericParameterizedSoilCarbonSmoothReservoirModel,
%           bgc_md2.resolve.mvars.NumericParameterizedVegetationCarbonSmoothReservoirModel,
%           bgc_md2.resolve.mvars.NumericSimulationTimes,
%           bgc_md2.resolve.mvars.NumericSoilCarbonMeanAgeSolutionArray,
%           bgc_md2.resolve.mvars.NumericSoilCarbonMeanBackwardTransitTimeSolution,
%           bgc_md2.resolve.mvars.NumericSoilCarbonStartMeanAgeTuple,
%           bgc_md2.resolve.mvars.NumericSolutionArray,
%           bgc_md2.resolve.mvars.NumericStartMeanAgeTuple,
%           bgc_md2.resolve.mvars.NumericStartValueArray,
%           bgc_md2.resolve.mvars.NumericStartValueDict,
%           bgc_md2.resolve.mvars.NumericVegetationCarbonMeanAgeSolutionArray,
%           bgc_md2.resolve.mvars.NumericVegetationCarbonMeanBackwardTransitTimeSolution,
%           bgc_md2.resolve.mvars.NumericVegetationCarbonStartMeanAgeTuple,
%           bgc_md2.resolve.mvars.OutFluxesBySymbol,
%           bgc_md2.resolve.mvars.OutputTuple,
%           bgc_md2.resolve.mvars.QuantityModelRun,
%           bgc_md2.resolve.mvars.QuantityParameterization,
%           bgc_md2.resolve.mvars.QuantityParameterizedSmoothReservoirModel,
%           bgc_md2.resolve.mvars.QuantitySimulationTimes,
%           bgc_md2.resolve.mvars.QuantitySolutionArray,
%           bgc_md2.resolve.mvars.QuantityStartValueArray,
%           bgc_md2.resolve.mvars.QuantityStartValueDict,
%           bgc_md2.resolve.mvars.SoilCarbonSmoothModelRun,
%           bgc_md2.resolve.mvars.SoilCarbonStateVariableTuple,
%           bgc_md2.resolve.mvars.StartConditionMaker,
%           bgc_md2.resolve.mvars.StateVarUnitTuple,
%           bgc_md2.resolve.mvars.StateVariableTuple,
%           bgc_md2.resolve.mvars.TimeSymbol,
%           bgc_md2.resolve.mvars.VegetationCarbonCompartmentalMatrix,
%           bgc_md2.resolve.mvars.VegetationCarbonInFluxesBySymbol,
%           bgc_md2.resolve.mvars.VegetationCarbonInputPartitioningTuple,
%           bgc_md2.resolve.mvars.VegetationCarbonInputScalar,
%           bgc_md2.resolve.mvars.VegetationCarbonInputTuple,
%           bgc_md2.resolve.mvars.VegetationCarbonInternalFluxesBySymbol,
%           bgc_md2.resolve.mvars.VegetationCarbonOutFluxesBySymbol,
%           bgc_md2.resolve.mvars.VegetationCarbonSmoothModelRun,
%           bgc_md2.resolve.mvars.VegetationCarbonStateVariableTuple,
%           inspect._empty,
%           sympy.physics.units.quantities.Quantity})
%\end{lstlisting}
%%%%%%%%%%%%%%%%%%%%%%%%%%%%%%%%%%%%%%%%%%%%%%%%%%%%%%%%%%%%%%%%%%%%%%%%%%%%%%%%%%%%%%%%%%%%%%%%%%
\section{Derivation of Matrix equations}
\label{appendix:MatrixDerivation}
Mathematically Compartmental Models are most economically described as graphs, where the set of compartments $\mathcal{P}$ and the set of non-negative fluxes $\mathcal{F}$ form the the nodes and edges respectively. 
Choosing one of $n!$ possible ways to enumerate  the set of pools $\mathcal{P}=\{p_0,\dots,p_n\}$ where $p_0$ is the outside world, we write the contents of the pools as $x_i \text{ for } i \in \{1,\dots,n\} $ and the fluxes as
\begin{align*}
%\X  &=(x_1,\dots x_n)^T
%\\
\mathcal{F} &=
\{
I_{0 \rightarrow j} > 0
\text{ for } j \in \{1,\dots ,n\}
\}  
\\
&
\cup
\{
F_{i \rightarrow j} > 0
\text{ for } i \in \{1,\dots ,n\} 
\text{ and } j \in \{1,\dots ,n\} j\ne i
\}
\text{ with }
F_{i \rightarrow j}=0 \text{ for }  x_{i} = 0 
\}
\\
&
\cup
\{
F_{i \rightarrow 0} 
\text{ for } i \in \{1,\dots ,n\} 
\text{ with }
F_{i \rightarrow 0}=0 \text{ for }  x_{i} = 0 
\}
\end{align*}
\label{massbalance} 
%back into a set of equations for single fluxes from which
%it was originally derived \cite{Jacquez1972} and 
where 
$ 
I_{0 \rightarrow j} 
$
are influxes from the outside into the system 
$
F_{i \rightarrow j} 
$
are fluxes between pools 
and 
$
F_{i \rightarrow 0} 
$
are fluxes out of the system.

In general all fluxes can depend on all the $x_i$  and time $t$ (trough environmental factors like  Temperature $T(t)$ and moisture $W(t)$.

The influxes don't have to depend on the $x_i$  but internal and out fluxes must at least depend on their source pool content to guarantee the 
condition that there is no outflux from an empty pool. 
% $
% F_{i \rightarrow * }=0 \text{ for }  \X_{i} = 0 
% $.
% Where we used the $*$ to indicate either another pool or the outside.
\newcommand{\xnt}{(x_1, \dots, x_n, t)}
For every pool we have a mass balance equation.
\begin{align}
  \frac{d}{d t} x_i 
    &= 
    \sum_{j\ne i} (-F_{i\rightarrow j}\xnt
    +F_{j\rightarrow j}\xnt ) 
    + I_{0 \rightarrow i}\xnt 
    - F_{i \rightarrow 0} \quad \forall i \in \{1,\dots n\}
\end{align}
Assuming continuity of the fluxes with respect to their source pool 
$F \in \mathcal C^1$ we can write them in product form. 
$F_{i \rightarrow j} = r_{j,i} x_i \text{ for } i \in \{1, \dots n\} , j \in \{ 0, 1,\dots ,n\} \text{ and } j \ne i $ 
\begin{align}
  \frac{d}{d t} x_i 
    &= - \underbrace{
      \left(
      r_{i \rightarrow } 
      + 
      \sum_{j \ne i} r_{j,i}\xnt
      \right)
      }_{=m_{i,i}\xnt}
      x_i
      +
      \sum_{j \ne i} \underbrace{r_{i,j}\xnt}_{- m_{i,j} \xnt } x_j
      +
      \underbrace{F_{\rightarrow i}\xnt}_{I_{\rightarrow i}}
    \\
    &= 
      -\sum_{j} m_{i,j}\xnt x_j + I_{\rightarrow i}\xnt
\end{align}
Writing 
$\X=(x_1,\dots x_n)^T$ for the ordered tuple of all pool contents, and $\I=(I_{\rightarrow 1},\dots I_{\rightarrow n})^T$ for the ordered tuple of all influxes, we get
\begin{align}
  \frac{d}{d t} \X &= \I(\X,t) - M(\X,t) \X \label{massbalance_0}
\end{align}
$-M$ is called the Compartmental Matrix. \footnote{Because the enumeration of the set of pools is arbitrary there are, for a model with $n$ pools actually $n!$ such matrix equations, that all describe the same model.}  

\subsubsection{Matrix decomposition} 
Together with a start-value $\X_0$ \eqref{massbalance_0}  constitutes an "initial value problem" (ivp) which can be solved numerically by moving step by step forward in time.

%Note: 
%
%It is mathematical standard notation to use $X$ in the *formulation* of the ivp (representing the momentary value) althoug *after we have solved it* the solution is expressed as function of time $X(t)$. This avoids confusion since everything appering with arguments is recognizable as explicitly calculable *before* we have solved the ivp.

Without further assumptions the system is "nonautonomous" (since either of $\I$ or $M$ can depend on time $t$) 
and "nonlinear" since either $M$ can depend on $\X$ or $\I$ can depend on $\X$ in a way that cannot be expressed in the form $\I(\X,t)=\tilde{\I(t)}+I_{mat}(t)\X$ with the matrix $I_{mat}(t)$ independent of $X$.

If $m_{i,i}(\X,t) \ne 0$ 
\footnote{
  If $m_{i,i}(\X,t) = 0 $ for some $i$ then some elements of $A$ become
  undefined. However, this does not mean that we could not write $M$ as a
  product, just that $A$ cannot be inferred and we have to pretend to know the
  $a_{j,i}\xnt \text{ for } j\ne i \text{ and } \forall \xnt \text{ with }
  k_{ii}\xnt=0 $ although we could never learn them from any observed fluxes.
  The same arguments holds for $\bv$.
}
it is possible to factorize $M(X,t)$ into a product $M=A(\X,t) K(\X,t)$ where $K$ is
a diagonal matrix and the matrix $A$ has only ones on it's main diagonal. 

If $u=\sum_{k=1\dots n} \I_k \ne 0$ it is possible to determine the dimensionless vector $\bv = \I/u$ where $\sum_{k=1\dots n} \beta_k =1$ and write $\I(\X,t)=\bv(\X,t)u(\X,t)$ 
Using these terms  we arrive at 
\begin{align*}
\frac{d \X}{d t}&=B(\X,t) u(\X,t) - A(\X,t) K(\X,t) \X   
\end{align*}
\newcommand{\kiixt}{
      \left(
      r_{i \rightarrow } \xnt
      + 
      \sum_{l \ne i} r_{l,i} \xnt
      \right)
}
with:
\begin{align}
  k_{i,i}\xnt &=\kiixt \nonumber
  \\
  a_{j,i}\xnt
  &=\frac{r_{j,i}\xnt}{k_{i,i}\xnt}=
  \left\{
  \begin{matrix}
    =\frac{
    r_{i,j}\xnt 
  }{
    \kiixt
  } \text{ for } j \ne i
  \\
  1 \text{ for } j=i
  \end{matrix}
  \right.
  \label{aij}
\end{align}
The $k_{i,i}$ can be interpreted as the rate of the total flux out of pool $i$. The elements of column $i$ of $A$ describe then which fractions of this total outflux is transferred to pool $j$. 

\subsubsection{Assumption of Linearity}
If we assume the model to be linear and nonautonomous the dependency on $\X$ vanishes and we have
either
\begin{align}
\frac{d \X}{d t}
  &=\tilde{\mathbf{I}}(t)+I_{mat}(t)\X - M(t) \X  \nonumber
  \\
  &=\underbrace{\tilde{\mathbf{I}}(t)+(I_{mat}(t) - M(t))}_{L(t)} \X \label{massbalance_linear} \\
  &=\tilde{\mathbf{I}}(t)+L(t) \X \nonumber
\end{align}
or if we insist on a non-state-dependent inputs 
\begin{align}
\frac{d \X}{d t}
  &=\I(t) - M(t) \X \label{massbalance_linear_no_state_dependent}
  \\
  &= \bv(t)u(t) - A(t) K(t) \X \nonumber.
\end{align} 
Eq. \eqref{mass-balance_linear} allows for influxes to be dependent on the receiving pool, e.g. for the influx of carbon through photosynthesis to depend on the the size of the leaf pool. Note that $L$ does not have to be compartmental and therefore not factorizable into $A$ and $K$.
Eq. \eqref{massbalance_linear_no_state_dependent} is that 
Both  exclude certain compartmental models e.g. some with interactions between chemical species. 
Imagine that some of the pools contain Nitrogen and others Carbon.
It is likely that some fluxes out of carbon pools are controlled by the
available Nitrogen.  
Imagine a compartmental system where the startvector
consist of Carbon and Nitrogen pool contents: $\X=(c_1,c_2,\dots,
n_1,n_2,\dots )^T$, then a flux between carbon pools $a$ and $b$ that
depends of the content of nitrogen pool $c$ depends on (a part of) the
statevector, which makes it nonlinear.
\begin{align*}
F_{a \rightarrow b} (\X,t)  &= r_{c_i \rightarrow *}(n_c,t) \X_a \\
                            &= r_{c_i \rightarrow *}(\X,t) \X_a
\end{align*}


\subsubsection{Assumption of Factorizability, substrate centered versus flux centered description}
For many published models the nonautonomous part  can be further localized into a diagonal matrix $\xi(t)$ so that we can achieve constant $A$ and $K$. It is important to realize two points here:
\begin{enumerate}
\item \label{substrate_xi}
  This is not possible for all compartmental matrices.
\item  \label{define_xi}
  In the cases where it is possible it does not uniquely define $\xi$.
\end{enumerate}

We can discuss (\ref{substrate_xi}) from a mathematical and a modeling viewpoint:
\newcommand{\kiit}{
      \left(
      r_{i \rightarrow } (t)
      + 
      \sum_{l \ne i} r_{l,i} (t)
      \right)
}
The linear version of \eqref{aij} is: 
\begin{align}
  k_{i,i}(t) &=\kiit \nonumber
  \\
  a_{j,i}(t) &=\left\{
  \begin{matrix}
  \frac{
    r_{j,i} (t)
  }{
    \kiit
  } \text{ for } j \ne i
  \\
  1 \text{ for } j=i
  \end{matrix}
  \right.
  \label{aij}
\end{align}

From this representation it is clear that the $a_{i,j}$ are only constant if all rates $r_{j,i} \text{ for } j \in \{0,\dots ,n \}$ contain the \emph{same} time dependent factor $\xi(t)$ , which makes the existence of constant $A$ and $K$ 
an \emph{assumption}.
From a modeling point of view the $\xi_{i,i}$ can be seen as a ``substrate'' dependent rate modifier since it affects everything that leaves the same pool in the same way, whereas $r_{i,*}(t)$ is specific to a single flux an so could be different for different ``processes'' even if they use the same substrate.


In order to discuss (\ref{define_xi}) we note that the assumption that we can write 
$M=A \xi K$ implies that we can also write it as $M=A \tilde{\xi} \tilde{K}$
where $\tilde{\xi}=d\xi$ , $\tilde{K}=d^{-1} K$ for any diagonal matrix $d$.
This implies that without further assumptions it is not possible to compute $\xi$
for a given model without a gauge condition like $\xi(T_0, W_0)=1$ for a some
specific temperature $T_0$ and moisture $W_0$, which in turn implies that the {\it baseline residence time } $(A K)^{-1}$
is only defined up to the above mentioned diagonal matrix $d$.
This fact becomes very important when certain properties of models are attributed to either $\xi$ or the {\it baseline residence time}.
Any sensible attribution of this kind has to be shown to be robust to changes of $d$.
\ref{xi_examples}
%%%%%%%%%%%%%%%%%%%%%%%%%%%%%%%%%%%%%%%%%%%%%%%
% Optional Glossary, Notation or Acronym section goes here:
%
% Glossary is only allowed in Reviews of Geophysics
%  \begin{glossary}
%  \term{Term}
%   Term Definition here
%  \term{Term}
%   Term Definition here
%  \term{Term}
%   Term Definition here
%  \end{glossary}


%%%%%%%%%%%%%%%%%%%%%%%%%%%%%%%%%%%%%%%%%%%%%%%
% Acronyms
%% NOTE that acronyms in the final published version will be spelled out when used in figure captions.
%   \begin{acronyms}
%   \acro{Acronym}
%   Definition here
%   \acro{EMOS}
%   Ensemble model output statistics
%   \acro{ECMWF}
%   Centre for Medium-Range Weather Forecasts
%   \end{acronyms}


%%%%%%%%%%%%%%%%%%%%%%%%%%%%%%%%%%%%%%%%%%%%%%%
% Notation
%   \begin{notation}
%   \notation{$a+b$} Notation Definition here
%   \notation{$e=mc^2$}
%   Equation in German-born physicist Albert Einstein's theory of special
%  relativity that showed that the increased relativistic mass ($m$) of a
%  body comes from the energy of motion of the body—that is, its kinetic
%  energy ($E$)—divided by the speed of light squared ($c^2$).
%   \end{notation}




%%%%%%%%%%%%%%%%%%%%%%%%%%%%%%%%%%%%%%%%%%%%%%%
%
% DATA SECTION and ACKNOWLEDGMENTS
%
%%%%%%%%%%%%%%%%%%%%%%%%%%%%%%%%%%%%%%%%%%%%%%%

\section*{Open Research Section}
The data used for the example could be obtained from several servers that offer TRENDYv9 data. \todo{Comment from Kostia: TRENDYv9 data is not publicly available \(unlike CMIP6\). We did get a password from the TRENDY group to access it. 
Besides, TRENDYv9 server is offline now, I suspect because they already released TRENDYv10 and are working on TRENDYv11. So, we should mention it that the example data can be obtained from the TRENDY group.
On the other hand, we can just add one more sentence that the codes are open source, that we aim at transparency and reproduceability and invite collaboration for the development of the packages. Maybe mention CCBY license, etc.
The remarks about the examples being not 'real' but 'reasonable' are important, but I would move them directly to the Examples section}
The example is however not intended to answer a `real' research question, but to show how a model comparison could look like,
if we \emph{had} access to the parameter sets used in the original trendy9 model runs.
We chose `reasonable' parameters for the models as \emph{if} we knew them for a pixel.
All code is available on GitHub and the example notebooks can be interactively explored on \href{https://mybinder.org/v2/gh/MPIBGC-TEE/bgc_md2/binder}{binder}.

%This section MUST contain a statement that describes where the data supporting the conclusions can be obtained. Data cannot be listed as ''Available from authors'' or stored solely in supporting information. Citations to archived data should be included in your reference list. Wiley will publish it as a separate section on the paper’s page. Examples and complete information are here:
%https://www.agu.org/Publish with AGU/Publishing/Author Resources/Data for Authors

% conventions for unique spelling
% fixpoint or fixed point (wikipedia)

\acknowledgments
Enter acknowledgments here. This section is to acknowledge funding, thank colleagues, enter any secondary affiliations, and so on.


%%%%%%%%%%%%%%%%%%%%%%%%%%%%%%%%%%%%%%%%%%%%%%%
% REFERENCES and BIBLIOGRAPHY
%
% \bibliography{<name of your .bib file>} don't specify the file extension
% don't specify bibliographystyle
%
%%%%%%%%%%%%%%%%%%%%%%%%%%%%%%%%%%%%%%%%%%%%%%%

%\bibliography{ enter your bibtex bibliography filename here }



%Reference citation instructions and examples:
%
% Please use ONLY \cite and \citeA for reference citations.
% \cite for parenthetical references
% ...as shown in recent studies (Simpson et al., 2019)
% \citeA for in-text citations
% ...Simpson et al. (2019) have shown...
%
%
%...as shown by \citeA{jskilby}.
%...as shown by \citeA{lewin76}, \citeA{carson86}, \citeA{bartoldy02}, and \citeA{rinaldi03}.
%...has been shown \cite{jskilbye}.
%...has been shown \cite{lewin76,carson86,bartoldy02,rinaldi03}.
%... \cite <i.e.>[]{lewin76,carson86,bartoldy02,rinaldi03}.
%...has been shown by \cite <e.g.,>[and others]{lewin76}.
%
% apacite uses < > for prenotes and [ ] for postnotes
% DO NOT use other cite commands (e.g., \citet, \citep, \citeyear, \nocite, \citealp, etc.).
%



\end{document}




More Information and Advice:

%%%%%%%%%%%%%%%%%%%%%%%%%%%%%%%%%%%%%%%%%%%%%%%
%
%  SECTION HEADS
%
%%%%%%%%%%%%%%%%%%%%%%%%%%%%%%%%%%%%%%%%%%%%%%%

% Capitalize the first letter of each word (except for
% prepositions, conjunctions, and articles that are
% three or fewer letters).

% AGU follows standard outline style; therefore, there cannot be a section 1 without
% a section 2, or a section 2.3.1 without a section 2.3.2.
% Please make sure your section numbers are balanced.
% ---------------
% Level 1 head
%
% Use the \section{} command to identify level 1 heads;
% type the appropriate head wording between the curly
% brackets, as shown below.
%
%An example:
%\section{Level 1 Head: Introduction}
%
% ---------------
% Level 2 head
%
% Use the \subsection{} command to identify level 2 heads.
%An example:
%\subsection{Level 2 Head}
%
% ---------------
% Level 3 head
%
% Use the \subsubsection{} command to identify level 3 heads
%An example:
%\subsubsection{Level 3 Head}
%
%---------------
% Level 4 head
%
% Use the \subsubsubsection{} command to identify level 3 heads
% An example:
%\subsubsubsection{Level 4 Head} An example.
%
%%%%%%%%%%%%%%%%%%%%%%%%%%%%%%%%%%%%%%%%%%%%%%%
%
%  IN-TEXT LISTS
%
%%%%%%%%%%%%%%%%%%%%%%%%%%%%%%%%%%%%%%%%%%%%%%%
%
% Do not use bulleted lists; enumerated lists are okay.
% \begin{enumerate}
% \item
% \item
% \item
% \end{enumerate}
%
%%%%%%%%%%%%%%%%%%%%%%%%%%%%%%%%%%%%%%%%%%%%%%%
%
%  EQUATIONS
%
%%%%%%%%%%%%%%%%%%%%%%%%%%%%%%%%%%%%%%%%%%%%%%%

% Single-line equations are centered.
% Equation arrays will appear left-aligned.

Math coded inside display math mode \[ ...\]
 will not be numbered, e.g.,:
 \[ x^2=y^2 + z^2\]

 Math coded inside \begin{equation} and \end{equation} will
 be automatically numbered, e.g.,:
 \begin{equation}
 x^2=y^2 + z^2
 \end{equation}


% To create multiline equations, use the
% \begin{eqnarray} and \end{eqnarray} environment
% as demonstrated below.
\begin{eqnarray}
  x_{1} & = & (x - x_{0}) \cos \Theta \nonumber \\
        && + (y - y_{0}) \sin \Theta  \nonumber \\
  y_{1} & = & -(x - x_{0}) \sin \Theta \nonumber \\
        && + (y - y_{0}) \cos \Theta.
\end{eqnarray}

%If you don't want an equation number, use the star form:
%\begin{eqnarray*}...\end{eqnarray*}

% Break each line at a sign of operation
% (+, -, etc.) if possible, with the sign of operation
% on the new line.

% Indent second and subsequent lines to align with
% the first character following the equal sign on the
% first line.

% Use an \hspace{} command to insert horizontal space
% into your equation if necessary. Place an appropriate
% unit of measure between the curly braces, e.g.
% \hspace{1in}; you may have to experiment to achieve
% the correct amount of space.


%%%%%%%%%%%%%%%%%%%%%%%%%%%%%%%%%%%%%%%%%%%%%%%
%
%  EQUATION NUMBERING: COUNTER
%
%%%%%%%%%%%%%%%%%%%%%%%%%%%%%%%%%%%%%%%%%%%%%%%

% You may change equation numbering by resetting
% the equation counter or by explicitly numbering
% an equation.

% To explicitly number an equation, type \eqnum{}
% (with the desired number between the brackets)
% after the \begin{equation} or \begin{eqnarray}
% command.  The \eqnum{} command will affect only
% the equation it appears with; LaTeX will number
% any equations appearing later in the manuscript
% according to the equation counter.
%

% If you have a multiline equation that needs only
% one equation number, use a \nonumber command in
% front of the double backslashes (\\) as shown in
% the multiline equation above.

% If you are using line numbers, remember to surround
% equations with \begin{linenomath*}...\end{linenomath*}

%  To add line numbers to lines in equations:
%  \begin{linenomath*}
%  \begin{equation}
%  \end{equation}
%  \end{linenomath*}



