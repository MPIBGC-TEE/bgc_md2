
Compartmental systems are a particular class of dynamical systems that describe
the flow of conserved quantities such as mass and energy through a network of
interconnected compartments.  The main purpose and greatest challenge of this
work is to make such models {\bf transparent} by  facilitating  comparisons
between them.  Firstly the scientific challenge is to find the common
diagnostics by which we can compare models. The technical challenge is how to
implement them in a way to make them applicable to many models and thirdly how
to describe models compactly enough to facilitate a maintainable collection of
reasonable size and flexible enough to formulate models in the many different
ways in which they appear in the literature.  To address the first question we
enhance the common diagnostics of pool contents and fluxes by the the
computations of age and transit time distributions for whole models and
especially subsystems like vegation or soil which is essential to be able to
compare models with conceptually different pools, i.e. two ecosystem models
will likely both have a vegetation part , however consist it might consist of two pools
(i.e. leaf and root) in one but three pools (i.e. leaf, wood, root) in the
other.  
To address the technical challenge we avoid duplication in two very
rigorous and novel ways: We provide common eqivalent building blocks for models
implemented in symbolic math via \texttt{sympy} as special types, and a set of
type annotated functions operating on these types as building blocks which can
be combined by a graph library to compute every possible result,
reachable by any recursive combination of these functions. 
This allows us to formulate models compactly and
flexibly in different equivalent ways i.e. via fluxes, flowdiagrams, or
matrices and at the same time to avoid the implementation of many of similar
functions for the same result.  Thus we can also not only query and compare
what is provided, as in the model description records of a conventional data
base, but also what is {\bf computable} from them.  The combination of these
capabilities is expressed in the package \texttt{bgc\_md2} (Biogeochemical
Model Database ), which apart from the tools mentioned above also represents a small
(30+) collection of models, motivated by our own work on the terrestrial carbon
cycle.
As a proof of concept of this approach we use \texttt{bgc\_md2} to reconstruct
four models of the trendy9 model intercomparison, symbolicly, guess some parameters
run them with the trendy9 driver data and compare them with respect to transient 
mean ages and transit times of carbon trough the vegetation and soil subsystems.
This presentation concentrates on the software and only links to the mathematical foundations
which are published already or in preparation.

%implemented analysis tools include symbolic computations  of model
%decomposition into subsystems, flowdiagrams, transformation and reformulation
%with respect to different building blocks e.g. matrix versus flux equations and
%also difficult to obtain numerical metrics to characterize timescales of mass
%flow in compartmental systems such as the age of the mass and transit time
%trough the entire system, subsystems like vegetation or soil, or individual
%compartments.
%The actual computations were outsourced into other packages where \LAPM (Linear
%Autonomous Pool Models) provides functions for the analysis of compartmental
%systems at equilibrium, \CompartmentalSystems tools for the analysis of
%non-autonomous linear compartmental systems and \ComputabilityGraphs
%determines the set of all possible computations that can be performed on a
%model depending on information available from its description.
