
Compartmental systems are a particular class of dynamical systems that describe
the flow of conserved quantities such as mass and energy through a network of
interconnected compartments.  The main purpose and greatest challenge of this
work is to make such models {\bf transparent} by  facilitating  comparisons
between them.  The scientific challenge is to find the common
diagnostics by which we can compare models. The technical challenges are,
firstly to implement the diagnostics in a way that makes them applicable to all collected
models, secondly how to \emph{describe} models comparably to allow queries but flexibly enough 
to formulate them in the many different
ways in which they appear in the literature and thirdly compactly enough to facilitate a maintainable collection of reasonable size 

We enhance the common diagnostics of pool contents and fluxes by the 
computations of age and transit time distributions for whole models and
especially subsystems like vegation or soil which is essential 
to compare models with conceptually different pools, i.e. two ecosystem models
w.r.t the vegetation part , consisting however of two pools
(i.e. leaf and root) in one but three pools (i.e. leaf, wood, root) in the
other.  
We provide an extensible, declarative domain specific language (DSL) to enable (data base) queries and reduce the amount of model specific code by orders of magnitude. 
avoiding duplication in two novel ways: It defines an extensible set of 
building blocks common to all models and implemented in symbolic math via \texttt{sympy} as special types, and an extensible set of type annotated functions operating on these types as 
building blocks which can be combined automatically by a graph library to compute every result,
reachable by any recursive combination of these functions. 
This allows us to formulate models compactly and
flexibly in different equivalent ways i.e. via fluxes, flowdiagrams, or
matrices and at the same time to avoid the implementation of many similar
functions for the same result.  Thus we can also not only query and compare
what is \emph{provided}, as in the model description records of a conventional data
base, but also what is \emph{computable} from them.  

Apart from the technical aspects the rigorous description of models via a
strictly typed functional DSL also informs the choice of diagnostics variables by
excluding ambigously defined candidates that have been proposed in the literature. 
The combination of these capabilities is implemented in open source python
packages \texttt{bgc\_md2} (Biogeochemical Model Database ), 
\LAPM (Linear Autonomous Pool Models),
\CompartmentalSystems  and \ComputabilityGraphs, available on GitHub and for testing even without installation on binder. 
We proof the above mentioned concepts by comparing four trendy9 models with respect to transient ages and transit times.
%implemented analysis tools include symbolic computations  of model
%decomposition into subsystems, flowdiagrams, transformation and reformulation
%with respect to different building blocks e.g. matrix versus flux equations and
%also difficult to obtain numerical metrics to characterize timescales of mass
%flow in compartmental systems such as the age of the mass and transit time
%trough the entire system, subsystems like vegetation or soil, or individual
%compartments.
%The actual computations were outsourced into other packages where \LAPM (Linear
%Autonomous Pool Models) provides functions for the analysis of compartmental
%systems at equilibrium, \CompartmentalSystems tools for the analysis of
%non-autonomous linear compartmental systems and \ComputabilityGraphs
%determines the set of all possible computations that can be performed on a
%model depending on information available from its description.
